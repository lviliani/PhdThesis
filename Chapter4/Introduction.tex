\section{Introduction}
%%%%%%%%%%%%%%%%%%%%%%%%%%%%%%%%%%%%%%%%%%%%%%%%%%%%%%%%%%%%%%%%%%%%%%
\label{sec:Introduction}

In this chapter the measurement of the transverse momentum spectrum of the Higgs boson, produced in proton-proton collisions at a center-of-mass energy of $\sqrt{s}=8$\TeV, is reported.
This measurement can be used to directly inspect the perturbative QCD theory in the Higgs sector.
In particular the $p_T^H$ variable is sensitive to the Higgs production mode and the differential distribution in this variable can be used to inspect the effects of the top quark mass in the gluon fusion top loop. Moreover, any observed deviation from the SM expectation, especially in the tail of the $p_T^H$ distribution, could be a hint of physics beyond the SM.\\
Similar measurements have already been performed by CMS and ATLAS experiments in the ZZ and $\gamma\gamma$ Higgs decay channels.
The measurement reported here is the first measurement of the Higgs \pt spectrum in the WW decay channel.\\
The cross section has been measured in a fiducial phase space defined using generator level variables in order to mimic the experimental acceptance and reduce the systematic uncertainties on the procedure of extrapolating the results in a larger phase space.\\
The Higgs transverse momentum has been reconstructed calculating the vector sum of the dilepton system transverse momentum plus missing transverse energy 
\begin{equation}
\vec{p}_\mathrm{T}^\mathrm{H} = \vec{p}_\mathrm{T}^{\ell\ell} + \vec{p}_\mathrm{T}^\mathrm{miss}
\end{equation}
The signal has been extracted subtracting all backgrounds by means of a binned Maximum Likelihood fit and has been then corrected for the efficiency of the analysis selections and for the detector resolution effects using an unfolding procedure.\\
The differential measurement has been performed in six bins of \pth with variable widths, chosen to have approximately the same purity in each bin, as explained in section \ref{sec:AnalysisStrategy}.\\



