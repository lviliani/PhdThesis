\section{Analysis Strategy}
%%%%%%%%%%%%%%%%%%%%%%%%%%%%%%%%%%%%%%%%%%%%%%%%%%%%%%%%%%%%%%%%%%%%%%
\label{sec:AnalysisStrategy}

The analysis presented here is based on that used in the previously published \hwwllvv{}
measurements by CMS~\cite{Chatrchyan:2013iaa}, modified to be inclusive in the number of jets. 
This modification significantly reduces the uncertainties related to the modelling of the number of jets produced in association with the Higgs boson.


\subsection{Event reconstruction and selections}\label{sec:Selections}

%%% Physics objects definition
The electron selection is based on two multivariate discriminants, one specialised in identifying the electron object and the other for isolation. The cut value for each discriminant is optimised to provide a good fake electron rejection and to improve the signal acceptance.

Muons are reconstructed using the standard CMS selection and are required to be identified both in the tracker (\textit{tracker muon}) and in the muon chambers (\textit{global muon}). Additionally quality criteria on the muon track are required, such as to have at least 10 hits in the tracker (at least one of which in the pixel detector) and to have $\chi^2/ndf < 10$.
Muon isolation is based on the Particle-Flow algorithm. An MVA approach is considered, based on the radial distributions of the Particle-Flow candidates inside a cone of radius 0.5 around the muon direction.

The efficiencies for the identification and isolation of the electrons and muons are measured in data and in simulation selecting a pure sample of leptons coming from the Z$\to\ell\ell$ decay. The measured efficiencies are used as scale factors to correct the MC simulation to precisely model the data. Similarly, the trigger efficiency extracted fromdata is applied toMC samples to correct for the additional loss.

Jets in this analysis are reconstructed by combining the energy measured in the calorimeters and tracks from charged particles on basis of the standard CMS particle flow algorithm and using the anti-$k_T$ clustering algorithm with $\mathrm{R} = 0.5$. Events will be classsified into zero jet, one jet and VBF topologies by counting jets within $|\eta| < 4.7$ and for $\pt > 30$\GeV.

\textcolor{red}{Here I could add some details about jets (see AN/2012-194) if they are not already discussed in the objects section.}

Background events from \ttbar and single-top production are rejected applying a soft-muon veto and b-tagging veto. The former selection requires that in the event there are no muons from b-decays passing the following cuts: 
\begin{itemize}
\item the muon is reconstructed as TrackerMuon (and passes the TMLastStationTight ID);
\item the number of hits of the muon in the Silicon Tracker is greater than 10;
\item the transverse impact parameter of the muon is less than 0.2 cm;
\item if $\pt > 20$\GeV then the muon is required to be non-isolated with $ISO/\pt > 0.1$.

The latter veto rejects events that contain jets tagged as b-jets using two different algorithms for high and low \pt jets. For jets with \pt between 10 and 30 GeV, the Track-Counting-High-Efficiency (TCHE) algorithm, with a cut at 2.1 on the discriminating variable, is applied.
For jets above 30 GeV, a more performant algorithm, Jet-Probability (JP), is used. Jets are identified as b-jets by the JP algorithm if the discriminating variable has a value above 1.4.
In the following a b-tagged jet is defined as a jet, within $|\eta|<2.4$ (b-tagging requires the tracker information), with a value of the discriminating variable above the mentioned thresholds for the two algorithms.  


%%% Event selection
The event selection consists of several steps. The first step is to select \WW -like events applying a selection that is heavily based on the main analysis selection except for few different cuts explained below.
The \WW -like event preselection consists of the following set of cuts:
\begin{enumerate}
\item {\bf Lepton preselection}:
  \begin{itemize}
  \item at least two opposite-sign and opposite-flavour ($e\mu$) leptons reconstructed in the event;
  \item $|\eta|<2.5$ for electrons and $|\eta|<2.4$ for muons;
  \item $\pt>20~\GeV$ for the leading lepton. For the trailing lepton, the transverse momentum is required to be larger than 10~\GeV.
  \end{itemize}
\item {\bf Extra lepton veto}: the event is required to have two and only two opposite-sign leptons passing the lepton selection.
\item {\bf \MET preselection}: particle flow \MET is required to be greater than $20$\GeV.
\item {\bf Di-lepton mass cut}: $m_{\ell\ell} > 12$\GeV in order to reject low mass resonances and QCD backgrounds.
\item {\bf Di-lepton $p_T$ cut}: $p_T^{\ell\ell} > 30$\GeV.
\item {\bf projected \MET selection}: minimum projected \MET required to be larger than 20~\GeV.
\item {\bf Transverse mass}: $m_T^H>60$\GeV to reject Drell-Yan to $\tau\tau$ events. 
\end{enumerate}
In addition to the \WW-like preselection other cuts are applied in order to reduce the top background (\ttbar ans single-top), which is one of the main backgrounds in this final state. We operate two different selections depending on the number of jets with $p_T > 30$~\GeV in the event. This is done to suppress the top background both in the low $p_T^H$ region, where 0-jets events have the biggest contribution, and for higher values where also larger jet multiplicity events are important.
The selection for 0-jets events relies on a soft muon veto, which rejects events with non-isolated soft muons (likely belonging to b-jets), and on a soft jets (with $p_T < 30$~\GeV) anti b-tagging requirement.
The latter requirement exploits the Track Counting High Efficiency tagger (TCHE) to reject soft jets that are likely to come from b quarks hadronization.
These are exactly the same requirements applied in the 0-jets bin of the main analysis.

For events with a jet multiplicity greater or equal than one, we apply a different selection with respect to the main analysis. In this case we exploit the good b-tagging performances of the \textit{JetBProbability} tagger to reject all the jets with $p_T > 30$~\GeV that are likely to come from a b quark. This jet veto relies on a cut on the \textit{JetBProbability} tagger discriminant as has been also done in the VH (\hwwllnn) analysis \cite{CMS_PAS_HIG_13-017}. Any jet with a discriminant value below $1.4$ is identified as a non b-jet. The analysis selection requires no b-tagged jets with $p_T > 30$~\GeV.

\begin{figure}[b]
\centering
\includegraphics[width=0.8\textwidth]{images/cutflow2.pdf}
\caption{Effect of single selections on MC samples. The signal (red line) is multiplied by 100 and superimposed on stacked backgrounds. In each bin, corresponding to a different selection, is reported the expected number of events in MC at a luminosity of $19.46~\mathrm{fb}^{-1}$.\label{fig:cutflow}}
\end{figure}

A  cut-flow plot is reported in figure \ref{fig:cutflow} showing the effect of each selection on top of Monte Carlo samples. In the first bin, labelled as \textit{No cut}, no selection has been applied and the bin content correspond to the total expected number of events with a luminosity of $19.46~\mathrm{fb}^{-1}$. All the events in this bin have at least two leptons with a loose transverse momentum cut of $8$~\GeV. In the following bin the lepton cuts are applied, including the requirement to have two opposite-sign and opposite-flavour leptons and the extra lepton veto. Then are progressively reported all the other selections, showing the effect of each cut on backgrounds and signal. For each selection is also reported the expected signal over background ratio which after the full selection reach a maximum value around $3\%$.

\subsection{Fiducial phase space}
The Higgs boson transverse momentum is measured in a fiducial phase space, which is defined at generator level requiring
\begin{itemize}
%\item Exactly two status 3 leptons, an electron and a muon, with opposite charge, with $|\eta|<2.5$ and $\ptlmax>18~\mathrm{GeV}$ and $\ptlmin>8~\mathrm{GeV}$. No Additional status 3 lepton of any \pt
\item Exactly two status 3 leptons, an electron and a muon, originated from the \hwwllnn decays, with opposite charge, with $|\eta|<2.5$ and $\pt>20~\mathrm{GeV}$ and $\pt>10~\mathrm{GeV}$ for the leading and subleading leptons respectively.%No Additional status 3 lepton of any \pt
\item Generator level invariant mass of the two leptons $\mll>12~\mathrm{GeV}$.
%\item Vector sum of the two status 3 neutrinos $\ptvv>15~\mathrm{GeV}$.
\item Vector sum of the two status 3 leptons $\ptll>30~\mathrm{GeV}$.
\item Generator level transverse mass $\sqrt{(\ptll+\pt^{\nu\nu})^2 - (\vec{\ptll}+\vec{\pt^{\nu\nu}})^2}>50~\mathrm{GeV}$.
\end{itemize}

Experimentally, the Higgs boson transverse momentum is reconstructed as the vector sum of the lepton momenta in the transverse plane and \MET.
\begin{equation}
\vec{p}_\mathrm{T}^\mathrm{\,H} = \vec{p}_\mathrm{T}^{\,\ell\ell} + \vec{p}_\mathrm{T}^\mathrm{\,miss}
\end{equation}
Compared to other differential analysis of the Higgs cross section, such as those in the ZZ and $\gamma\gamma$ decay channels, this analysis has to cope with the limited resolution due to the \MET entering the transverse momentum measurement.
The effect of the limited \MET resolution has two main implications on the analysis strategy:
\begin{itemize}
\item the choice of the binning in the transverse momentum spectrum needs to be reasonable when compared to the resolution. A detailed explanation of how the binning is defined is given in Sec.~\ref{sec:Binning}.
\item Non negligible bin migration effects are present, and an unfolding procedure needs to be applied, not only to correct for selection efficiencies, as in ZZ and $\gamma\gamma$, but also to correct for bin migration effects. This is explained in Sec.~\ref{sec:Unfolding}.
\end{itemize}

A detailed description of the fiducial region definition and about its optimization is given in appendix \ref{app:fiducial_region}.


\subsection{Binning of the \pth distribution}

\textcolor{red}{Put here also the plots with efficiency and fakes in each \pth bin}.










