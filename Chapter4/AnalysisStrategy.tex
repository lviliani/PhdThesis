\section{Analysis Strategy}
%%%%%%%%%%%%%%%%%%%%%%%%%%%%%%%%%%%%%%%%%%%%%%%%%%%%%%%%%%%%%%%%%%%%%%
\label{sec:AnalysisStrategy}
The Higgs boson transverse momentum is measured in a fiducial phase space, which is defined at generator level requiring
\begin{itemize}
%\item Exactly two status 3 leptons, an electron and a muon, with opposite charge, with $|\eta|<2.5$ and $\ptlmax>18~\mathrm{GeV}$ and $\ptlmin>8~\mathrm{GeV}$. No Additional status 3 lepton of any \pt
\item Exactly two status 3 leptons, an electron and a muon, originated from the \hwwllnn decays, with opposite charge, with $|\eta|<2.5$ and $\pt>20~\mathrm{GeV}$ and $\pt>10~\mathrm{GeV}$ for the leading and subleading leptons respectively.%No Additional status 3 lepton of any \pt
\item Generator level invariant mass of the two leptons $\mll>12~\mathrm{GeV}$.
%\item Vector sum of the two status 3 neutrinos $\ptvv>15~\mathrm{GeV}$.
\item Vector sum of the two status 3 leptons $\ptll>30~\mathrm{GeV}$.
\item Generator level transverse mass $\sqrt{(\ptll+\pt^{\nu\nu})^2 - (\vec{\ptll}+\vec{\pt^{\nu\nu}})^2}>50~\mathrm{GeV}$.

\end{itemize}
Experimentally, the Higgs boson transverse momentum is reconstructed as the vector sum of the lepton momenta in the transverse plane and MET.
\begin{equation}
\vec{p}_T^H = \vec{p}_T^{\ell\ell} + \vec{p}_T^{miss}
\end{equation}
Compared to other differential analysis of the Higgs cross section, such as those in the ZZ and $\gamma\gamma$ decay channels, this analysis has to cope with the limited resolution due to the \MET entering the transverse momentum measurement.
The effect of the limited \MET resolution has two main implications on the analysis strategy:
\begin{itemize}
\item the choice of the binning in the transverse momentum spectrum needs to be reasonable when compared to the resolution. A detailed explanation of how the binning is defined is given in Sec.~\ref{sec:Binning}.
\item Non negligible bin migration effects are present, and an unfolding procedure needs to be applied, not only to correct for selection efficiencies, as in ZZ and $\gamma\gamma$, but also to correct for bin migration effects. This is explained in Sec.~\ref{sec:Unfolding}.
\end{itemize}

A detailed description of the fiducial region definition and about its optimization is given in appendix \ref{app:fiducial_region}.

The selection is essentially based on the one in the \hwwllnn published analysis \cite{Chatrchyan:2013iaa} with one noticeable difference being the fact that in this analysis we do not make categories in the number of jets. The reason for this choice is that the number of jets is strongly correlated with the transverse momentum, so making an inclusive analysis in the number of jets allows the dropping of most of the uncertainties related to the signal modeling of the number of jets produced in association with the Higgs boson. A detailed description of the selection is shown in Sec.~\ref{sec:Selections}.

The estimation of the backgrounds is different, to some extent, with respect to the one of the published \hwwllnn. This is mainly due to the absence of the jet binning. The techniques used to assess the backgrounds in each bin are discussed in Secs.~\ref{sec:TTBackground}, \ref{sec:WWBackground}, \ref{sec:OtherBackgrounds}.

Concerning the signal extraction, this analysis is again based on the already published \hwwllnn analysis, although we fit the signal component in each of the transverse momentum bins, using two dimensional templates in the \mll, \mt plane. The signal extraction is discussed in Sec.~\ref{sec:SignalExtraction}. 

Finally an unfolding procedure is needed to extract the differential distribution in a fiducial phase space. This is discussed in detail in Sec.~\ref{sec:Unfolding}.

