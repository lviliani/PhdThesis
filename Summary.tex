\chapter*{Summary}
\addcontentsline{toc}{chapter}{Summary}  
\markboth{Summary}{}
\thispagestyle{empty}

This thesis work has been organized according to a twofold purpose, reporting firstly a precision measurement of the Higgs boson transverse momentum spectrum using proton-proton collision data collected at a centre-of-mass energy of 8\TeV and thereafter focusing on the first data collected by the CMS experiment at the unprecedented centre-of-mass energy of 13\TeV.

The transverse momentum spectrum and the inclusive cross section times branching fraction for the Higgs boson production have been reported using $\rm{H}\to\rm{W^+W^-}\to e^{\pm} \mu^{\mp}\nu\nu$ decays. Measurements have been performed using data from proton-proton collisions at a centre-of-mass energy of 8\TeV collected by the CMS experiment at the LHC and corresponding to an integrated luminosity of 19.4\ifb. The Higgs boson transverse momentum has been reconstructed using the lepton pair transverse momentum and missing transverse momentum. A two-dimensional template fit based on dilepton invariant mass and transverse mass has been used to extract the signal contribution. The differential cross section times branching fraction has been measured as a function of the Higgs boson transverse momentum in a fiducial phase space defined to match the experimental acceptance in terms of lepton kinematics and event topology. An unfolding procedure has been used to extrapolate the measured results to the fiducial phase space and to correct for the detector effects.
The measurements have been compared to SM theoretical estimations provided by the \textsc{HRes} and \textsc{Powheg V2} generators, showing good agreement within the experimental uncertainties. The inclusive cross section times branching fraction in the fiducial phase space has been measured to be $39\pm 8~(\mathrm{stat}) \pm 9~(\mathrm{syst})\,\mathrm{fb}$, consistent with the SM expectation.

The first 13\TeV proton-proton collision data collected during 2015, corresponding to an integrated luminosity of 2.3\ifb, have been used to perform a re-discovery analysis of the Higgs boson in the $\rm{H}\to\rm{W^+W^-}\to e^{\pm} \mu^{\mp}\nu\nu$ decay channel. The signal strength has been measured performing a two-dimensional template fit using the dilepton invariant mass and transverse mass to separate signal and background contributions.
The observed signal strength, which is driven by the low data statistics, has been found to be $0.3\pm0.5$, corresponding to an observed significance of $0.7\,\sigma$, to be compared with the expected value of $2.0\,\sigma$ for a Higgs boson mass of 125\GeV.

The same 13\TeV data have been used to search for new resonances decaying to $\rm{W^+W^-}\to e^{\pm} \mu^{\mp}\nu\nu$ in the mass range between 200\GeV and 1\TeV. The analysis relied on a maximum likelihood template fit using a transverse mass variable that is able to discriminate between signal and background contributions. No significant excess with respect to the SM background expectation has been observed, and exclusion limits on the production cross section times branching fraction of the new resonance have been reported over the whole analysed mass spectrum, studying also different hypotheses of the resonance decay width.
