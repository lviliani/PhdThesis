\section{Systematic uncertainties}\label{chap6:Systematics}

The systematic uncertainties affecting this analysis are the same discussed in Sec.~\ref{chap5:systs}. The differences with respect to the Higgs boson cross section measurement presented in Chapter~\ref{chap5} are described below.

The PDF and $\alpha_s$ uncertainties on the signal cross sections are taken from the computations performed by the LHC cross section working group~\cite{YRtmp}, and are included for all the mass points. The value of these uncertainties depends on the resonance mass and vary from 3 to 5\% for ggH and from 2 to 3\% for VBF production modes. The PDFs and $\alpha_{s}$ uncertainties on the signal selection are evaluated for every resonance mass and are found to be less than 1\% for both ggH and VBF.

The theoretical uncertainties in the signal yields due to the jet categorization are evaluated for all the ggH signals following the prescription described in Refs.~\cite{Stewart:2011cf,Heinemeyer:2013tqa}.

An additional uncertainty on the modelling of the \ttbar background is derived from the observed discrepancy between data and \textsc{Powheg V2} plus \textsc{pythia 8.1} simulation on the top quark \pt spectrum~\cite{Khachatryan:2015oqa}, which is particularly important in the tail of the \mti distribution. Another uncertainty affecting the \mti tail for the top quark background is the parton shower uncertainty. This is evaluated comparing the generator level \mti distributions corresponding to two different simulations of the \ttbar process: one obtained using \textsc{pythia 8.1} for the showering and hadronization of the simulated events, and the other using \textsc{herwig++}. The difference between the two is used to extract a shape uncertainty, which is less than 1\% for low \mti values and reaches about 6\% in the \mti tail.






