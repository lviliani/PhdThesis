\section{Introduction}\label{chap6:introduction}

In this chapter, a search for a high mass spin-0 particle (from now on denoted as X) in the X$\rightarrow$WW$\rightarrow \ell\nu\ell'\nu'$ decay channel is presented, where $\ell$ and $\ell'$ refer to an different flavour lepton pair, i.e. e$\mu$. 
The search is based upon proton-proton collision data samples corresponding to an integrated luminosity
of up to 2.3\ifb  at $\sqrt{s} = 13$\TeV, recorded by the CMS experiment at the LHC during 2015. This analysis represents a general extension of the SM Higgs boson search presented in \ref{chap5} and is performed in a range of heavy scalar masses from $M_\mathrm{X} = 200$\GeV up to 1\TeV, extending the range studied in a similar analysis performed using Run 1 LHC data~\cite{Khachatryan:2015cwa}, which provided upper limits on the production cross section of new scalar resonances up to 600\GeV.

Despite the discovery of a particle consistent with the SM Higgs boson in 2012, there is a possibility that this particle is only a part of a larger Higgs sector, and hence only partially responsible of the EW symmetry breaking. This can be achieved in different theoretical models that extends the SM, such as the two-Higgs-doublet models~\cite{Branco:2011iw,craig,Haber:2015},
or models in which the SM Higgs boson mixes with a heavy EW singlet, which predict the existence
of an additional resonance at high mass, with couplings similar to those of the SM Higgs boson, as most recently described in~\cite{Chpoi:2013wga,Robens:2015gla}.

This analysis reports a generic search for a scalar particle with different resonance decay widths hypothesis, produced via the ggH and VBF production mechanisms. The results can then be interpreted in terms of different theoretical models. This analysis is heavily based on the SM Higgs search described in \ref{chap5}, in terms of physics objects, selections and background estimation. The differences and similarities are discussed in this chapter.
