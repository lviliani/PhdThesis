%\documentclass[a4paper,12pt,english,twoside,openright]{book}
\documentclass[draft,english,booktabs,hyperref,titling]{hepthesis}
%%% hepthesis requirements
\usepackage{setspace,fancyhdr,rotating,comment,tocbibind,caption,changepage,varwidth,csquotes,babel,a4wide,amsmath,hyperref,booktabs,draftcopy,lineno,titling}
%\usepackage[babel]{csquotes}
\usepackage{hepnames,hepunits,hepparticles}
\usepackage{epsfig,amssymb,latexsym,numprint,textcomp}
%\usepackage[utf8]{inputenc}
%\usepackage{appendix}
\usepackage{epstopdf}
\usepackage{subfigure}
\usepackage{graphicx}
\usepackage[backend=biber, sorting=none]{biblatex}
\usepackage[labelfont=bf]{caption}
\usepackage{enumerate}
\usepackage{lscape}
\usepackage{emptypage}
\usepackage{pifont}

\newcommand{\hww}{\ensuremath{\mathrm{H}\to\mathrm{WW}}\xspace}
\newcommand{\hwwllnn}{\ensuremath{\mathrm{H}\to\mathrm{WW}\to\mathrm{2\ell2\nu}}\xspace}
\newcommand{\fb}{\ensuremath{\mathrm{fb}}}
\newcommand{\pb}{\ensuremath{\mathrm{pb}}}
\newcommand{\ifb}{\ensuremath{\mathrm{fb^{-1}}}}
\newcommand{\ipb}{\ensuremath{\mathrm{pb^{-1}}}}
\newcommand{\WW}{\ensuremath{\mathrm{WW~}}}
\newcommand{\W}{\ensuremath{\mathrm{W~}}}
\newcommand{\F}{\ensuremath{\cmsSymbolFace{F}}}
\newcommand{\V}{\ensuremath{\cmsSymbolFace{V}}\xspace}
\newcommand{\VVV}{\cmsSymbolFace{VVV}\xspace}
\newcommand{\Wjets}{\ensuremath{\PW\text{+jets}}\xspace}
\newcommand{\X}{\ensuremath{\cmsSymbolFace{X}}\xspace}
\newcommand{\delphill}{\ensuremath{\Delta\phi_{\ell\ell}}}
\newcommand{\delphillmet}{\ensuremath{\Delta\phi(\ell\ell,\VEtmiss)}}
\newcommand{\dyll}{\ensuremath{\cPZ/\gamma^*\to \ell^+\ell^-}\xspace}
\newcommand{\dymm}{\ensuremath{\cPZ/\gamma^*\to\Pgmp\Pgmm}}
\newcommand{\dytt}{\ensuremath{\cPZ/\gamma^* \to\tau^+\tau^-}}
\newcommand{\jp}{\ensuremath{J^{P}}\xspace}
\newcommand{\mll}{\ensuremath{m_{\ell\ell}}\xspace}
\newcommand{\mt}{\ensuremath{m_\mathrm{T}}\xspace}
\newcommand{\psvectorKD}{\ensuremath{\mathcal{D}_{1^-}}\xspace}
\newcommand{\ptl}{\ensuremath{p_\perp^{\ell}}\xspace}
\newcommand{\ptll}{\ensuremath{\pt^{\ell\ell}}\xspace}
\newcommand{\qq}{\ensuremath{\Pq\Pq}\xspace}
\newcommand{\superKD}{\ensuremath{\mathcal{D}_\text{bkg}} }
\newcommand{\tw}{\ensuremath{\cPqt\PW}\xspace}
\newcommand{\vectorKD}{\ensuremath{\mathcal{D}_{1^+}} }
\newcommand{\wgamma}{\ensuremath{\PW\gamma}\xspace}
\newcommand{\mjj}{\ensuremath{m_{jj}}\xspace}
\newcommand{\pt}{\ensuremath{p_\mathrm{T}}\xspace}
\newcommand{\pth}{\ensuremath{p_\mathrm{T}^\mathrm{H}}\xspace}




\pdfinfo{%
  /Title    (PhD Thesis)
  /Author   (Lorenzo Viliani)
}


\bibliography{biblio.bib}

\begin{document}

\begin{frontmatter}

\title{Measurement of the Higgs boson transverse momentum spectrum in the WW decay channel at 8\,TeV and first results at 13\,TeV}
\author{Lorenzo Viliani}

\titlepage[of University of Florence]{PhD Thesis}

\begin{abstract}
The cross section for Higgs boson production in pp collisions is studied using the $\mathrm{H} \to \mathrm{W}^+ \mathrm{W}^-$  decay mode, followed by leptonic decays of the W bosons, leading to an oppositely charged electron-muon pair in the final state. 
The measurements are performed using data collected by the CMS experiment at the LHC with pp collisions at a centre-of-mass energy of 8\TeV, corresponding to an integrated luminosity of 19.4\ifb.
The Higgs boson transverse momentum (\pt) is reconstructed using the lepton pair \pt and missing \pt. The differential cross section times branching fraction is measured as a function of the Higgs boson \pt in a fiducial phase space defined to match the experimental acceptance in terms of the lepton kinematics and event topology. The production cross section times branching fraction in the fiducial phase space is measured to be $39 \pm 8~$(stat)$~\pm 9~$(syst)\fb. The measurements are compared to theoretical calculations based on the standard model to which they agree within experimental uncertainties.
\end{abstract}

%\pagestyle{empty}
\cleardoublepage
%\pagenumbering{roman}
\pagestyle{fancy}
\fancyhead{}
\fancyfoot{}
\fancyhead[RO]{\thepage}
\fancyhead[LE]{\thepage}
\tableofcontents
\cleardoublepage

\end{frontmatter}

\begin{mainmatter}

\input Chapter1/chapter1.tex
\input Chapter2/chapter2.tex
\input Chapter3/chapter3.tex
\input Chapter4/chapter4.tex
\input Chapter5/chapter5.tex
\input Chapter6/chapter6.tex
\input Chapter7/chapter7.tex

\end{mainmatter}

\begin{backmatter}

\clearpage\thispagestyle{empty}\cleardoublepage
\addcontentsline{toc}{section}{\bibname}
\printbibliography
\clearpage\thispagestyle{empty}\cleardoublepage
%\pagestyle{empty}

\end{backmatter}

\end{document}
