\section{Systematic uncertainties}\label{chap5:systs}

The systematic uncertainties affecting this measurement can be divided into three categories: the uncertainties in the background estimation, experimental uncertainties and theoretical uncertainties.

The first category includes the uncertainties related to the background normalization and (\mll, \mt) shape. For the nonresonant WW production the (\mll, \mt) shape is taken from simulation. The input normalization to the fit is set to the expected value from simulation (scaled to match the NNLO cross section), and an unconstrained nuisance parameter with a flat prior distribution is associated to it, in order to freely float the normalization in the fit. This is done separately for the two jet categories.

The top quark background shape is taken from simulation after applying b tagging scale factors. The uncertainties in the normalization are treated similarly to the WW background case, but constraining the corresponding nuisance parameters by means of two control regions orthogonal to the signal phase space. A similar procedure is used for estimating the normalization of the \dytt background process.

Effects due to experimental uncertainties are studied by applying a scaling and smearing of variables related to the physics objects, e.g. the \pt of the leptons, followed by a subsequent recalculation of all the correlated variables. This is done for simulation to account for possible systematic mismodelling.

All experimental sources of uncertainty, except for the one related to luminosity, are treated both as normalization and shape uncertainties, and the correlations among signal and background processes in all categories are taken into account. The following experimental uncertainties are considered:
\begin{itemize}
\item the uncertainty determined by the CMS online luminosity monitoring, which is of 2.7\% for the first data collected at $\sqrt{s}=13$\TeV;
\item the acceptance uncertainty associated with the combination of single and double lepton triggers, which is 2\%;
\item the lepton reconstruction and identification efficiency uncertainties, that are in the range 0.5--5\% for electrons and 1--7\% for muons depending on \pt and $\eta$;
\item the muon momentum and electron energy scale and resolution uncertainties, that amount to 0.01--0.5\% for electrons and 0.5--1.5\% for muons depending on \pt and $\eta$;
\item the jet energy scale uncertainties, that vary between 1--11\% depending on the \pt and $\eta$ of the jet;
\item the \MET resolution uncertainty, that is taken into account by propagating the corresponding uncertainties in the leptons and jets;
\item the uncertainty in b tagging and mistag scale factors. These systematic uncertainties are anticorrelated between the top quark enriched control region and the other ones.
\end{itemize}

The uncertainties in the signal and background production rates due
to theoretical uncertainties include several components, which are assumed to be
independent: the PDFs and $\alpha_{s}$, the underlying event and parton shower model,
and the effect of missing higher-order corrections.

The effects of the variation of PDFs, $\alpha_s$ and renormalization/factorization QCD scales mainly affect the signal processes, since the most important backgrounds are estimated using data driven techniques. However, the uncertainties in minor backgrounds that are estimated from simulation are taken into account. These uncertainties are split into uncertainties in cross section, which are computed by the LHC cross section working group~\cite{deFlorian:2016spz}, and selection efficiency~\cite{Butterworth:2015oua}. The PDFs and $\alpha_{s}$ signal cross section  uncertainties are about 6--7\% for ggH and 1--3\% for VBF production mechanism. The PDFs and $\alpha_{s}$ acceptance uncertainties are less than 1\% for all gluon- and quark-induced processes. The effect of varying the renormalization and factorization scales on the selection efficiency is around 1--3\% depending on the specific process. 
%To estimate these uncertainties, the events are reweighted according to different QCD scales or different PDF sets and the selection efficiency is recomputed each time. For the QCD scale uncertainty the maximum variation with respect to the nominal value is taken as the uncertainty. For the case of PDF and $\alpha_s$ uncertainties, the distribution of the selection efficiency is built taking into account all the replicas in the NNPDF3.0 set and the uncertainty is estimated as the standard deviation of that distribution.

In addition, the categorization of events based on jet multiplicity introduces additional uncertainties in the ggH production mode related to missing higher order corrections. These uncertainties are evaluated following the prescription described in Refs.~\cite{Stewart:2011cf,Heinemeyer:2013tqa} and are found to be of 5.6\% for the 0 jets and 13\% for the 1 jet category.

The underlying event uncertainty in the signal contribution is estimated by comparing two different \textsc{Pythia 8} tunes, while parton shower modelling uncertainty is estimated by comparing samples interfaced with \textsc{Pythia 8} and \textsc{Herwig++} programs. 
The effect on the ggH (VBF) expected yield is about 5\% (5\%) for the \textsc{Pythia 8} tune variation and about 7\% (10\%) for the parton shower description.

Other specific theoretical uncertainties are associated to some backgrounds. An uncertainty in the ratio of the \ttbar and tW cross sections is included. Indeed, these two processes are characterized by a different number of b-jets in the final state (2 b-jets for \ttbar and 1 for tW) and the b veto acts differently for the two. A variation of the relative ratio of the cross sections can thus cause a migration of events from the 0 to the 1 jet categories and viceversa. The uncertainty in the \ttbar/tW cross section ratio is 8\%, according to the theoretical cross section calculations~\cite{Aliev:2010zk,Beneke:2011mq}.

For what the $\mathrm{q\bar{q}'\to WW}$ background shape is concerned, an uncertainty related to the diboson \pt reweighting is evaluated varying the renormalization, factorization and resummation QCD scales.

Finally, the uncertainties due to the limited statistical accuracy of the MC simulations are also taken into account, including an independent uncertainty for each bin of the two-dimensional  (\mll, \mt) distribution, and for each category. The uncertainty for a certain bin and process is given by the standard deviation of the Poisson distribution with mean corresponding to the number of simulated events in that bin.

 
