\section{Introduction}\label{sec:chap5_introduction}

In this chapter, the first analysis of the \hww decay at 13\TeV is presented, using a total integrated luminosity of 2.3\ifb,
collected during the 2015 proton proton running period of the LHC.

Final states in which the two \W bosons decay leptonically are studied.
Therefore, events with a pair of oppositely-charged leptons,
exactly one electron and one muon, a substantial amount of missing transverse energy, \MET, 
due to the presence of neutrinos in the final state, and either zero or one jet are selected. 
This signature is common to other processes, which enter the analysis as backgrounds.
The main background comes from WW production, irreducible background that shares the same final states and can only 
be separated by the use of certain kinematic properties.
Another important background is W+jets, where a jet can mimick a leptonic signature.
Background coming from top quark events, i.e. \ttbar and single top production, is also important, 
followed by other processes such as Drell-Yan, WZ, and other EWK production.
The analysis strategy follows the one used during Run 1 in the same channel, described in Chapter~\ref{chap4}, with a few different aspects that are described in the next sections.

With respect to 8\TeV, the ggH production cross section at 13\TeV is expected to increase of a factor of 2, thus raising the number of expected signal events. In addition, the cross section for the background processes is increasing as well. The WW production cross section increases of a factor of 1.8 and and the \ttbar cross section of a factor of 3.5, due to the enhancement of the gluon PDFs at higher center of mass energies. 
