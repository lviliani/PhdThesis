\section{Analysis strategy}\label{chap5:analysis_strategy}

Regarding the electrons, muons, jets and \MET definition and reconstruction, the standard CMS recommendations described in Chapter~\ref{chap2} are used. The specific selections used in this analysis are briefly summarised below.

Muons are identified according to the CMS recommendations for the medium working point, with the addition of some extra cuts, as defined by the following selections:
\begin{itemize}
\item identified by the standard medium muon selection described in Sec.~\ref{sec:Objects}; \textcolor{red}{Not yet defined :)}
\item $\pt> 10$\GeV;
\item $|\eta < 2.4|$;
\item $|d_{xy}| < 0.01$\,cm for $\pt < 20$\GeV and $|d_{xy}| < 0.02$\,cm for $\pt > 20$\GeV, $d_{xy}$ being the transverse impact parameter with respect to the primary vertex;
\item $|d_{z}| < 0.1$\,cm, where $d_z$ is the longitudinal distance of the muon track in the tracker extrapolated along the beam direction.
\end{itemize}

For the muon isolation the CMS recommended particle flow isolation based on the tight working point is used, corresponding to a requirement on the isolation variable of $ISO_\mathrm{tight} < 0.15$. In addition a tracker relative isolation is also applied.

For the electron identification, the tight working point is used. In addition some additional cuts to make the selection ``trigger-safe'' are included. This is done because the electron triggers already include some identification and isolation requirements that are based on the raw detector information, while the offline selections makes use of particle flow requirements. The ``trigger-safe'' selections are defined to make the the offline identification and isolation requirements tighter with respect to the online triggers.

The simulated events are corrected for the lepton trigger, identification and isolation efficiencies measured in data using the same techniques described in Sec.~\ref{sec:Selections}.
