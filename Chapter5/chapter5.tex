%\chapter[Search for the SM Higgs boson in the \boldmath$\hww$ channel with the first \boldmath$13\TeV$ LHC data]{Search for the SM Higgs boson in the \boldmath$\hww$ channel with the first \boldmath$13\TeV$ LHC data}\label{chap5}
\chapter[Measurement of Higgs boson production using \boldmath$\hww$ decays with first \boldmath$13\TeV$ data]{Measurement of Higgs boson production using \boldmath$\hww$ decays with first \boldmath$13\TeV$ data}\label{chap5}
\chaptermark{Measurement of Higgs boson production using \boldmath$H\to W^+W^-$ decays at \boldmath$13\TeV$}
\thispagestyle{empty}

In this chapter, the first measurement of the 125\GeV Higgs boson cross section times branching ratio to a W boson pair at 13\TeV is presented. The analysis makes use of a total integrated luminosity of 2.3\ifb, collected during the 2015 proton-proton data taking period of the LHC. The analysis strategy follows the one described in Chapter~\ref{chap4}, with some differences described in the following.

With respect to the centre-of-mass energy of 8\TeV, the ggH production cross section at 13\TeV is expected to increase of a factor of 2, thus raising the number of expected signal events. Nevertheless, the cross section of the background processes increases as well: in particular, the WW production cross section of a factor of 1.8 and the \ttbar cross section of a factor of 3.5.

%\section{Introduction}
%%%%%%%%%%%%%%%%%%%%%%%%%%%%%%%%%%%%%%%%%%%%%%%%%%%%%%%%%%%%%%%%%%%%%%
\label{sec:Introduction}
The discovery of a new boson consistent with the standard model (SM) Higgs boson has been reported by ATLAS and CMS Collaborations in 2012.
The discovery has been followed by a comprehensive set of studies of properties of this new boson in several production and decay channels and no evidence of deviation from the SM expectation has been found so far. The CMS studies in the \hwwllnn decay channel include the measurement of the Higgs properties, as well as constraints on the Higgs total decay width and gauge bosons anomalous couplings.

In this section the measurement of the transverse momentum spectrum of the Higgs boson, produced in proton-proton collisions at a center-of-mass energy of $\sqrt{s}=8$\TeV, is reported.
This measurement can be used to directly inspect the perturbative QCD theory in the Higgs sector.
In particular the $p_T^H$ variable is sensitive to the Higgs production mode and the differential distribution in this variable can be used to inspect the effects of the top quark mass in the gluon fusion top loop. Moreover, any observed deviation from the SM expectation, especially in the tail of the $p_T^H$ distribution, could be a hint of physics beyond the SM.\\
Similar measurements have already been performed by CMS and ATLAS experiments in the ZZ and $\gamma\gamma$ Higgs decay channels.
The measurement reported here is the first measurement of the Higgs \pt spectrum in the WW decay channel.\\
The cross section has been measured in a fiducial phase space defined using generator level variables in order to mimic the experimental acceptance and reduce the systematic uncertainties on the procedure of extrapolating the results in a larger phase space.\\
The Higgs transverse momentum has been reconstructed calculating the vector sum of the dilepton system transverse momentum plus missing transverse energy 
\begin{equation}
\vec{p}_\mathrm{T}^\mathrm{H} = \vec{p}_\mathrm{T}^{\ell\ell} + \vec{p}_\mathrm{T}^\mathrm{miss}
\end{equation}
The signal has been extracted subtracting all backgrounds by means of a binned Maximum Likelihood fit and has been then corrected for the efficiency of the analysis selections and for the detector resolution effects using an unfolding procedure.\\
The differential measurement has been performed in six bins of \pth with variable widths, chosen to have approximately the same purity in each bin, as explained in section \ref{sec:AnalysisStrategy}.\\




\section{Data and simulated samples}\label{chap5:dataset}

Data recorded in proton-proton collisions at 13\TeV during 2015 are used in the analysis, with a total integrated luminosity of 2.3\ifb. Single and double lepton triggers are used, similarly to the same analysis at 8\TeV. The HLT trigger \pt thresholds used in this analysis are listed in Table~\ref{tab:trigger13TeV}.

\begin{table}[htb]
\begin{center}
\caption{Transverse momentum thresholds required for lepton triggers at HLT level. 
         Double set of thresholds indicates the thresholds for each leg of the double lepton triggers.}
\begin{tabular}{ccc}
\toprule
Trigger path       & Threshold \\
\midrule
Single electron    & $\pt > 23$\GeV         \\  
Single muon        & $\pt > 20$\GeV         \\ 
Muon-Electron      & $\pt > 17$ and $12$\GeV         \\ 
Electron-Muon      & $\pt > 17$ and $8$\GeV         \\ 
\bottomrule
\end{tabular}
\label{tab:trigger13TeV} 
\end{center}
\end{table}

\begin{comment}
\begin{table}[h]
\caption{HLT paths related to Electrons}
\label{table:ele_trigg_13}
\scalebox{0.75}{
\begin{tabular}{ll}
\hline
HLT Path & Description \\
\hline\hline
HLT\_Ele23\_WPLoose\_Gsf\_v*   &
\parbox{11cm}{$\,$ \\Single Electron trigger. Best trigger to be used for 2015 data. In HWW, we are using ``Trigger safe'' Id. Turn on is at around Ele $\rm p_{T}$ = 30 GeV\\}\\
\hline
HLT\_Ele17\_Ele12\_CaloIdL\_TrackIdL\_IsoVL\_DZ\_v* &
\parbox{11cm}{$\,$ \\Double Electron Trigger. Best trigger to cover the turn on region from single electron trigger. ``DZ'' filter is also present. Its efficiency is also calculated separately.}\\
\hline
HLT\_Ele12\_CaloIdL\_TrackIdL\_IsoVL\_v* &
\parbox{11cm}{$\,$ \\This electron leg of \\
HLT\_Mu17\_TrkIsoVVL\_Ele12\_CaloIdL\_TrackIdL\_IsoVL\_v*\\
same as Ele12 leg of double electron trigger.\\} \\
\hline
HLT\_Ele17\_CaloIdL\_TrackIdL\_IsoVL\_v*&
\parbox{11cm}{$\,$ \\This electron leg of\\
HLT\_Mu8\_TrkIsoVVL\_Ele17\_CaloIdL\_TrackIdL\_IsoVL\_v* \\
same as Ele17 leg of double electron trigger.} \\
\hline
\end{tabular}
}
\end{table}

\begin{table}
\caption{Muon trigger's elements description}
\label{table:mu_trigg_13}
\scalebox{0.8}{
\begin{tabular}{ll}
\hline
HLT path \\
\hline\hline
HLT\_IsoMu18\_v*   & 
\parbox{11cm}{$\,$ \\single muon trigger\\}\\
\hline
HLT\_IsoTrMu20\_v* &
\parbox{11cm}{$\,$ \\single muon trigger with tracker isolation\\}\\
\hline
HLT\_Mu17\_TrkIsoVVL & 
\parbox{11cm}{$\,$ \\leg for the HLT\_Mu17\_TrkIsoVVL\_Mu8\_TrkIsoVVL\_DZ\_v*,\\
HLT\_Mu17\_TrkIsoVVL\_TkMu8\_TrkIsoVVL\_DZ\_v* and\\ 
HLT\_Mu17\_TrkIsoVVL\_Ele12\_CaloIdL\_TrackIdL\_IsoVL\_v*\\
double lepton triggers\\} \\
\hline
HLT\_Mu8\_TrkIsoVVL &
\parbox{11cm}{$\,$ \\leg for the HLT\_Mu17\_TrkIsoVVL\_Mu8\_TrkIsoVVL\_DZ\_v* and\\
HLT\_Mu8\_TrkIsoVVL\_Ele17\_CaloIdL\_TrackIdL\_IsoVL\_v* \\ 
double lepton triggers\\} \\
\hline
HLT\_TkMu8\_TrkIsoVVL &
\parbox{11cm}{$\,$ \\leg for the HLT\_Mu17\_TrkIsoVVL\_TkMu8\_TrkIsoVVL\_DZ\_v*\\
double muon trigger\\} \\
\hline
$DZ_{\mu\mu}$ &
\parbox{11cm}{$\,$ \\efficiency of DZ cut in \\
the HLT\_Mu17\_TrkIsoVVL\_Mu8\_TrkIsoVVL\_DZ\_v*\\
and HLT\_Mu17\_TrkIsoVVL\_TkMu8\_TrkIsoVVL\_DZ\_v* \\
double muon triggers, it is around 95\%\\} \\
\hline
\end{tabular}
}
\end{table}
\end{comment}

The trigger efficiencies are measured in data and applied to simulated events as described in Sec.~\ref{subsec:Datasets}.

%Monte Carlo
Concerning the simulated samples, several different MC generators are used. 
Higgs boson signal samples are simulated in all channels using \textsc{Powheg v2}~\cite{Nason:2004rx,Frixione:2007vw,Alioli:2010xd}, designed to describe the full NLO properties of these processes.
In particular, for Higgs boson produced via ggH~\cite{Alioli:2008tz}, and VBF~\cite{Nason:2009ai},
the decay into two W bosons and subsequently into leptons is done using \textsc{JHUGen} v5.2.5. 
For associated production with a vector boson ($\mathrm{W}^{+}\mathrm{H}$, $\mathrm{W}^{-}\mathrm{H}$, ZH)~\cite{Luisoni:2013kna}, including gluon fusion produced ZH (ggZH), 
the Higgs boson decay is instead simulated using \textsc{pythia} 8.1~\cite{Sjostrand:2007gs}. All the signal samples are generated assuming a Higgs boson mass of 125\GeV.

The \textsc{Powheg v2}~\cite{Melia:2011tj} is also used for simulating the $\mathrm{q\bar q}$ induced WW  production in different decay channels. The simulated events are reweighted to reproduce the $\pt^\mathrm{WW}$ distribution obtained from \pt-resummed calculations~\cite{Meade:2014fca,Jaiswal:2014yba}. Gluon fusion produced WW is generated at LO QCD accuracy using \textsc{mcfm} v7.0~\cite{Campbell:2013wga}.
%The cross section used for normalizing WW processes produced via $\mathrm{q\bar q}$ was computed at next-to-next-to-leading order (NNLO)~\cite{Gehrmann:2014fva}. 
%In order to control the top quark background processes, the analysis is performed with events that have no more than one high-\pt jet. The veto on high-\pt jets enhances the importance of logarithms of the jet \pt, spoiling the convergence of fixed-order calculations of the qq$\rightarrow$WW process and requiring the use of dedicated resummation techniques for an accurate prediction of differential distributions~\cite{Meade:2014fca,Jaiswal:2014yba}. The \pt of the jets produced in association with the WW system is strongly correlated with its transverse momentum, $\pt^\mathrm{WW}$, especially in the case where only one jet is produced. The simulated qq$\rightarrow$WW events are reweighted to reproduce the $\pt^\mathrm{WW}$ distribution from the \pt-resummed calculation.

The \ttbar process with dilepton final state is also generated using \textsc{Powheg v2}. The simulated processes for the WW and \ttbar production are illustrated in Table~\ref{tab:wwl}, together with the associated cross sections. Other minor background samples are also generated, a list of the most relevant ones is presented in Table~\ref{tab:otherbck}.

\begin{table}[htb]
\caption{Simulated processes for \ttbar and \WW production.}\label{tab:wwl}
\begin{center}
\begin{tabular}{lc}
\toprule
Process & $\sigma\times\mathcal{B}$ [pb] \\
\midrule
\ttbar$\rightarrow$WW$\mathrm{b\bar{b}}\rightarrow2\ell2\nu \mathrm{b\bar{b}}$ & 87.31 \\
$\mathrm{q\bar q}\rightarrow$WW$\rightarrow2\ell2\nu$ & 12.178 \\
$\mathrm{gg}\rightarrow$WW$\rightarrow2\ell2\nu$ & 0.5905 \\
%$gg\rightarrow$\WW$\rightarrow2\ell2\nu$ (H diagr.) & 0.9544\\
\bottomrule
\end{tabular}
\end{center}
\end{table}

\begin{table}[htb]
\caption{Simulated samples for other minor background processes used in the analysis. Single top quark production includes the dominant tW process, as well as the sub-dominant production in the s- and t-channels.\label{tab:otherbck}}
\begin{center}
\begin{tabular}{lc}
\toprule
Process & $\sigma\times\mathcal{B}$ [pb] \\
\midrule
Single top &   71.7  \\
Drell-Yan ($10\GeV < \mll < 50\GeV$)  &  20471.0  \\
Drell-Yan ($\mll > 50\GeV$)   &  6025.26  \\
WZ$\to2\ell2\mathrm{q}$ &  5.5950 \\
ZZ$\to2\ell2\mathrm{q}$ &  3.2210 \\
WWZ &  0.1651 \\
WZZ &  0.05565 \\
ZZZ &  0.01398  \\
\bottomrule
\end{tabular}
\end{center}
\end{table}

All processes are generated using NNPDF2.3~\cite{Ball:2013hta,Ball:2011uy} for NLO generators.
The LO version of the same PDF set is used for LO generators. All the event generators are interfaced  to \textsc{pythia} 8.1~\cite{Sjostrand:2007gs} for the showering of
partons and hadronization, as well as including a simulation of the underlying event (UE) and multiple interaction (MPI) based on the CUET8PM1 tune~\cite{Khachatryan:2015pea}. 

To estimate the systematic uncertainties related to the choice of UE and MPI tune, the signal processes and the WW background are also generated with two alternative tunes which are representative of the errors on the tuning parameters.
The showering and hadronization systematic uncertainty is estimated by interfacing the same MC samples with the \textsc{herwig++} 2.7 parton shower~\cite{Richardson:2013nfo,Bellm:2013hwb} instead of \textsc{pythia 8}.

Drell-Yan (DY) production of Z/$\gamma^{*}$ is generated using \textsc{amc@nlo}~\cite{Alwall:2014hca}. 
Other multiboson processes, such as WZ,ZZ, and VVV (V$=$W/Z), are generated with \textsc{amc@nlo} and normalized to the cross section calculation at NLO accuracy.

The simulated samples are generated with distributions for the number of pile-up interactions that are meant to roughly cover, though not exactly match, the conditions expected for the different data-taking periods. In order to factorize these effects, the number of true pile-up interactions from the simulation truth is reweighted to match the data.
In Fig.~\ref{Fig:pu}, the effect of this reweighting on a sample enriched in Drell-Yan events is shown. The average number of pile-up is approximately $11.5$.

\begin{figure}[htbp]
\centering
\includegraphics[width=0.45\textwidth]{images/13TeV/nvertices.png}
\caption{
    Distribution of the number of vertices in a Drell-Yan enriched phase space in data,
    together with the simulation before (red) and after (solid green) the pile-up reweighting.}
    \label{Fig:pu}
\end{figure}

For Higgs boson signal, the inclusive cross sections used are the ones reported by the LHC Higgs Cross Section Working Group~\cite{temphiggsxsecs}. The ggH cross section is computed at NNLO+NNLL QCD and NLO EW accuracy, while NNLO QCD and NLO EW accuracy is used for the other production modes. The branching fractions are the ones reported in Ref.~\cite{Heinemeyer:2013tqa}.

The cross section used for the $\mathrm{q\bar q}$ induced WW processes is computed at NNLO QCD accuracy~\cite{Gehrmann:2014fva}. The normalization of this background is eventually directly taken from a fit to data and the NNLO cross section is used just as an initial guess. The LO cross section for the gluon induced WW process is obtained directly from \textsc{mcfm}, and a $k$-factor of 1.4 is applied to correct for the difference between the LO and NLO theoretical calculation~\cite{Caola:2015rqy}.
The contribution of the interference between the $\mathrm{gg \to WW}$ and $\mathrm{gg\to H\to WW}$ processes is also evaluated using \textsc{mcfm} and is found to be negligible compared to the signal contribution.

The cross sections of the different single top processes are estimated by the LHC Top Working group~\cite{singletop} with NLO accuracy.
The \ttbar cross section is also provided by the LHC Top Working group~\cite{topxsec}, and it is computed at NNLO accuracy, with NNLL soft gluon resummation.


\section{Analysis strategy}\label{chap5:analysis_strategy}

Regarding the electrons, muons, jets and \MET definition and reconstruction, the standard CMS recommendations described in Chapter~\ref{chap2} are used. The specific selections used in this analysis are briefly summarised below.

Muons are identified according to the CMS recommendations for the medium working point, with the addition of some extra cuts, as defined by the following selections:
\begin{itemize}
\item identified by the standard medium muon selection described in Sec.~\ref{sec:Objects}; \textcolor{red}{Not yet defined :)}
\item $\pt> 10$\GeV;
\item $|\eta < 2.4|$;
\item $|d_{xy}| < 0.01$\,cm for $\pt < 20$\GeV and $|d_{xy}| < 0.02$\,cm for $\pt > 20$\GeV, $d_{xy}$ being the transverse impact parameter with respect to the primary vertex;
\item $|d_{z}| < 0.1$\,cm, where $d_z$ is the longitudinal distance of the muon track in the tracker extrapolated along the beam direction.
\end{itemize}

For the muon isolation the CMS recommended particle flow isolation based on the tight working point is used, corresponding to a requirement on the isolation variable of $ISO_\mathrm{tight} < 0.15$. In addition a tracker relative isolation is also applied.

For the electron identification, the tight working point is used. In addition some additional cuts to make the selection ``trigger-safe'' are included. This is done because the electron triggers already include some identification and isolation requirements that are based on the raw detector information, while the offline selections makes use of particle flow requirements. The ``trigger-safe'' selections are defined to make the the offline identification and isolation requirements tighter with respect to the online triggers.

The simulated events are corrected for the lepton trigger, identification and isolation efficiencies measured in data using the same techniques described in Sec.~\ref{sec:Selections}.

\section{Background estimation}\label{chap5:backgrounds}

The main background processes affecting the analysis signature, non-resonant WW production and top quark processes, are estimated using data. Backgrounds arising from an experimental misidentification of the objects, such as W+jets (also called ``Fake''), are estimated using data as well. The other minor backgrounds are generally estimated directly from simulation as described in the following subsections.

\subsection{WW background}

The quark-induced WW background is simulated with NLO accuracy in perturbative QCD, and the transverse momentum of the diboson system is reweighted to match the NNLO+NNLL accuracy from theoretical calculations~\cite{Meade:2014fca,Jaiswal:2014yba}. However, given the large uncertainties on the jet multiplicity distribution associated to this process, the normalization of this background is measured from data separately for the 0 and 1 jet categories. The normalization k-factors are extracted directly from the fit together with the signal strengths, leaving the WW normalization free to float separately in the two jet multiplicity categories. An orthogonal control region for the WW background normalization estimation is not needed in this case, owing to the different $m_{\ell\ell}$-$\mt$ shape for signal and background.

The gluon-induced WW production is sub-dominant with respect to the quark-induced production, and its shape and normalization is fully taken from simulation, scaling the cross section to the theoretical prediction with NLO accuracy~\cite{Caola:2015rqy}.

\subsection{Top quark background}

As explained in Sec.~\ref{chap5:analysis_strategy}, the production of top quark pairs represents one of the dominant backgrounds in this analysis given its large cross section and a similar final state compared to the signal. A b-jet veto, based on the \emph{cMVAv2} b tagging algorithm, is used to suppress this background and a reweighting procedure is applied on top of the simulated events to correct for different b tagging efficiency in data and simulation.

The top quark background normalization is measured using data, defining a b-jets enriched control region by inverting the b-jet veto. More precisely, the b-jets enriched control region for the 0-jet category is defined with the same WW baseline selection but requiring at least one jet with $20<\pt<30$\GeV to be identified as a b jet and no other jets with $\pt > 30$\GeV. For the 1-jet category, the b-jets enriched region is defined requiring exactly one jet with $\pt>30$\GeV identified as a b-jet.
To reduce other backgrounds in these two regions, the dilepton mass has to be greater than 50\GeV. Distributions of the \mll and \mt variables in the b-jets enriched control regions after applying the data driven estimation are shown in Figure~\ref{fig:TopCtrl}, for the 0 and 1 jet categories separately.

The top quark background normalization is constrained during the fit procedure separately in the two jet categories, by means of the control regions defined above, which are treated in the fit as two additional categories. 

\begin{figure}[htbp]
\centering
\includegraphics[width=0.45\textwidth]{images/13TeV/cratio_hww2l2v_13TeV_top_of0j_mll.png}
\includegraphics[width=0.45\textwidth]{images/13TeV/cratio_hww2l2v_13TeV_top_of0j_mth.png}
\includegraphics[width=0.45\textwidth]{images/13TeV/cratio_hww2l2v_13TeV_top_of1j_mll.png}
\includegraphics[width=0.45\textwidth]{images/13TeV/cratio_hww2l2v_13TeV_top_of1j_mth.png}
\caption{
Distributions of \mll (left) and \mt (right) for events with 0 jet (top) and 1 jet (bottom) 
in top enriched phase space.
Scale factors estimated from data are applied. The first (last) bin includes underflows (overflows).
}
\label{fig:TopCtrl}
\end{figure}

\subsection{Jet-induced (or Fake) background}

One of the primary source belonging to this category arises from the misidentification of leptons in W+jets processes in the 0 jet category. Also, semileptonic \ttbar decays contribute especially for higher jet multiplicities. Multijet production and hadronic \ttbar decays are also taken into account, but have a much smaller contribution.

This background is fully estimated using data, with the technique described in Sec.~\ref{sec:wjetsbkg}. To check the agreement of the background estimated in this way with data, a control sample enriched in jet-induced events is defined. The events in the control sample are selected applying the WW baseline requirements but requesting an e$\mu$ pair with same charge, which significantly suppresses the WW and \ttbar processes. The \mll distributions in this control region for the 0 and 1 jet categories are shown in Fig.~\ref{fig:13TeVsamesign}. From the crosscheck in this control region, a global normalization factor of 0.8 is derived and applied to the jet-induced background.

\begin{figure}[!h]
\centering
    \subfigure[0 jet]{
    \includegraphics[width=0.45\textwidth]{images/13TeV/cratio_hww2l2v_13TeV_ss_of0j_mll.png}
    }
    \subfigure[1 jet]{
    \includegraphics[width=0.45\textwidth]{images/13TeV/cratio_hww2l2v_13TeV_ss_of1j_mll.png}
    }
    \caption{
         Control plots for $m_{\ell\ell}$ in a fakes enriched phase space for events with 0 and 1 jet with $p_{T} > 30$~GeV,
         in e$\mu$ final state.
         Fake contribution has been scaled by 0.8 to match data.
         }\label{fig:13TeVsamesign}
\end{figure}



\subsection{DY background}\label{chap5:DYbackground}

This background contributes to the analysis phase space because of the $\mathrm{Z}/\gamma^*$ decays to a pair of $\tau$ leptons, which consequently decays to an e$\mu$ pair. This background process is predominant in the low \mt region, which is used as an orthogonal control region to determine the background normalization in the 0 and 1 jet categories separately. In particular this control region is defined by selecting events with $\mt < 60$\GeV and $30\GeV < \mll < 80$\GeV. The \mll distributions in these control regions for the 0 and 1 jet categories are shown in Fig.~\ref{fig:13TeVDYtt}.

As for the top quark background, the normalization of this background in the 0 and 1 jet categories, is constrained directly in the fit by means of the control regions, which are treated as two additional categories.

The kinematics of this background is taken from simulation, after reweighting the Z boson \pt spectrum to match the observed distribution measured in data. In fact, this variable is not well reproduced by the MC generator used for simulating this process, especially in the bulk of the distribution, the discrepancy being ascribed to the missing contribution from resummed calculations.

\begin{figure}[htbp]
\centering
\includegraphics[width=0.45\textwidth]{images/13TeV/cratio_hww2l2v_13TeV_dytt_of0j_mll.png}
\includegraphics[width=0.45\textwidth]{images/13TeV/cratio_hww2l2v_13TeV_dytt_of1j_mll.png}
\caption{
Distributions of \mll for events with 0 jet (left) and 1 jet (right) in the DY$\rightarrow \tau\tau$  enriched control region. Scale factors estimated from data are applied.
}
\label{fig:13TeVDYtt}
\end{figure}

\subsection{Other backgrounds}\label{chap5:otherBackgrounds}

The W$\gamma^*$ and the WZ electroweak processes can be gathered in the same physical process, although the final state kinematics is rather different. In particular, the invariant mass of the leptons arising from the $\gamma^*$ decays is generally below 4\GeV, while the leptons from the Z boson decay are characterized by a larger invariant mass. Another background which can be experimentally identical to those is the W$\gamma$ production, where a real photon is produced in association with a W boson and consequently undergoes a photon conversion to leptons due to the interaction with the material constituting the first layers of the silicon tracker.

All these backgrounds may contribute to the signal phase space whenever one of the three leptons escape from the detector acceptance or is not identified. The shape and cross section of these backgrounds are taken from simulation. The only exception is the normalization of the W$\gamma^*$ background, being this process dominant in the low \mll region, which is scaled to data defining a proper control region. The control region is defined selecting events with three isolated muons, with $\pt > 10$,5 and 3\GeV for the first three leading muons respectively. The selection is further defined by $\MET < 25\GeV$ and $\MET$ projected to the leading muon $<$ 45\GeV. The pair of muons with the smallest invariant mass is taken as coming from the $\gamma^{*}$ decay. The k-factor measured in data for this background to be applied in the simulation is $1.98\pm0.54$.

All remaining backgrounds from di-boson and tri-boson production, which are of minor importance in the analysis phase space, are normalized according to their expected theoretical cross sections.






\section{Systematic uncertainties}\label{chap5:systs}

The systematic uncertainties affecting this measurement can be divided into three categories: the uncertainties on the background estimation, experimental uncertainties and theoretical uncertainties.

The first category includes the uncertainties related to the background normalization and (\mll,\mt) shape. For the nonresonant WW production the (\mll,\mt) shape is taken from simulation. The input normalization to the fit is set to the expected value from simulation (scaled to match the NNLO cross section), and an unconstrained nuisance parameter with a flat prior distribution is associated to it, in order to freely float the normalization in the fit. This is done separately for the two jet categories.

The top quark background shape is taken from simulation after applying b tagging scale factors. The uncertainties on the normalization are treated similarly to the WW background case, but constraining the corresponding nuisance parameters by means of two control regions orthogonal to the signal phase space. A similar procedure is used for estimating the normalization of the \dytt background process.

Effects due to experimental uncertainties are studied by applying a scaling and smearing of variables related to the physics objects, e.g. the \pt of the leptons, followed by a subsequent recalculation of all the correlated variables. This is done for simulation to account for possible systematic mismodelling.

All experimental sources of uncertainty, except for the one related to luminosity, are treated both as normalization and shape uncertainties, and are correlated among the signal and background processes in all categories. The following experimental uncertainties are considered:
\begin{itemize}
\item the uncertainty determined by the CMS online luminosity monitoring, 2.7\% for the first data collected at $\sqrt{s}=13$\TeV;
\item the acceptance uncertainty associated with the combination of single and double lepton triggers, which is 2\%;
\item the lepton reconstruction and identification efficiency uncertainties, that are in the range 0.5--5\% for electrons and 1--7\% for muons depending on \pt and $\eta$;
\item the muon momentum and electron energy scale and resolution uncertainties, that amount to 0.01--0.5\% for electrons and 0.5--1.5\% for muons depending on \pt and $\eta$;
\item the jet energy scale uncertainties, that vary between 1--11\% depending on the \pt and $\eta$ of the jet;
\item the \MET resolution uncertainty, that is taken into account by propagating the corresponding uncertainties on the leptons and jets;
\item the uncertainty on b tagging and mistag scale factors. These systematic uncertainties are anticorrelated between the top quark enriched control region and the other ones.
\end{itemize}

The uncertainties in the signal and background production rates due
to theoretical uncertainties include several components, which are assumed to be
independent: the PDFs and $\alpha_{s}$, the underlying event and parton shower model,
and the effect of missing higher-order corrections.

The effects of the variation of PDFs, $\alpha_s$ and renormalization/factorization QCD scales, mainly affect the signal processes, since the most important backgrounds are estimated using data driven techniques. However, the uncertainties on minor backgrounds that are estimated from simulation are taken into account. These uncertainties are split into uncertainties on cross section, which are computed by the LHC cross section working group~\cite{YRtmp}, and selection efficiency~\cite{Butterworth:2015oua}. The PDFs and $\alpha_{s}$ signal cross section  uncertainties are about 6--7\% for ggH and 1--3\% for VBF production mechanism. The PDFs and $\alpha_{s}$ acceptance uncertainties are less than 1\% for all gluon- and quark-induced processes. The effect of varying the renormalization and factorization scales on the selection efficiency is around 1--3\% depending on the specific process. 
%To estimate these uncertainties, the events are reweighted according to different QCD scales or different PDF sets and the selection efficiency is recomputed each time. For the QCD scale uncertainty the maximum variation with respect to the nominal value is taken as the uncertainty. For the case of PDF and $\alpha_s$ uncertainties, the distribution of the selection efficiency is built taking into account all the replicas in the NNPDF3.0 set and the uncertainty is estimated as the standard deviation of that distribution.

In addition, the categorization of events based on jet multiplicity introduces additional uncertainties on the ggH production mode related to missing higher order corrections. These uncertainties are evaluated following the prescription described in Refs.~\cite{Stewart:2011cf,Heinemeyer:2013tqa} and are found to be of 5.6\% for the 0 jets and 13\% for the 1 jet category.

The underlying event uncertainty on the signal contribution is estimated by comparing two different \textsc{pythia 8} tunes, while parton shower modelling uncertainty is estimated by comparing samples interfaced with \textsc{pythia 8} and \textsc{herwig++} programs. 
The effect on the ggH (VBF) expected yield is about 5\% (5\%) for the \textsc{pythia 8} tune variation and about 7\% (10\%) for the parton shower description.

Other specific theoretical uncertainties are associated to some backgrounds. An uncertainty on the ratio of the \ttbar and tW cross sections is included. Indeed, these two processes are characterized by a different number of b-jets in the final state (2 b-jets for \ttbar and 1 for tW) and the b veto acts differently for the two. A variation of the relative ratio of the cross sections can thus cause a migration of events from the 0 to the 1 jet categories and viceversa. The uncertainty on the \ttbar/tW cross section ratio is 8\%, according to the theoretical cross section calculations~\cite{topxsec,singletop}.

%The $\mathrm{gg}\to\mathrm{WW}$ background LO cross section predicted by the \textsc{mcfm} generator is scaled to the NLO calculation, applying a k-factor of 1.4 with an uncertainty of 15\%~\cite{Caola:2015rqy}. The interference term between the $\mathrm{gg}\to\mathrm{WW}$ and the ggH signal is also included and simulated with LO accuracy using \textsc{mcfm}. The k-factor to scale the interference term is 1.87, given by the geometrical average of the LO to NNLO gg$\to$H$\to$WW scale factor (2.5) and the LO to NLO $\mathrm{gg}\to\mathrm{WW}$ scale factor (1.4). The uncertainty on this value is estimated as the maximum variation with respect to the two scale factors mentioned above, and is found to be of 25\%. Anyway, with the current amount of integrated luminosity, the interference contribution is found to be negligible.

For what the $\mathrm{qq}\to\mathrm{WW}$ background shape is concerned, an uncertainty related to the diboson \pt reweighting is evaluated varying the renormalization, factorization and resummation QCD scales.

Finally, the uncertainties due to the limited statistical accuracy of the MC simulations are also taken into account, including an independent uncertainty for each bin of the two-dimensional  (\mll,\mt) distribution, and for each category. The uncertainty for a certain bin and process is given by the standard deviation of the Poisson distribution with mean corresponding to the number of simulated events in that bin.

 

\section{Results}\label{chap5:results}

The expected and observed signal significance are shown in Table~\ref{tab:13TeVsignif} for all the categories separately. Also, the observed signal strengths and the corresponding uncertainties are shown. The best fit signal strength obtained combining all the categories together is found to be $0.3^{+0.5}_{-0.5}$, corresponding to an observed significance of $0.7\,\sigma$, to be compared with the expected significance of $2.0\,\sigma$ for a Higgs boson mass of 125\GeV.

\begin{table}
\caption{Observed and expected significance and signal strength  the SM Higgs boson with a mass of 125\GeV for the 0-jet and 1-jet, $\mu$e and e$\mu$, categories.}\label{tab:13TeVsignif}
\begin{center}
\begin{tabular}{lccc}
\toprule
Category  &  Expected significance      &  Observed  significance    &  $\sigma/\sigma_{SM}$     \\
\midrule
0-jet  $\mu$e   &     1.1        &  1.3        &  1.13 $_{-0.9}^{+0.9}$             \\ [5pt]   

0-jet  e$\mu$   &     1.3        &  0.4        &  0.33 $_{-0.7}^{+0.7}$             \\ [5pt]   

1-jet  $\mu$e   &     0.8        &  0          &  -0.11$_{-1.7}^{+0.5}$                 \\ [5pt] 

1-jet  e$\mu$   &     0.9        &  0          &  -0.54$_{-1.4}^{+1.4}$                 \\ [5pt] 

\midrule 

0-jet           &     1.6        &  1.3       &  0.71$_{-0.5}^{+0.6}$             \\ [5pt]  

1-jet           &     1.2        &  0         &  -0.56$_{-1.0}^{+1.0}$                \\ [5pt]  

\midrule 
Combination     &     2.0        &  0.7       &  0.33$_{-0.5}^{+0.5}$              \\ [5pt]  
\bottomrule
\end{tabular}
\end{center}
\end{table}


