\section{Beyond the Standard Model}
%%%%%%%%%%%%%%%%%%%%%%%%%%%%%%%%%%%%%%%%%%%%%%%%%%%%%%%%%%%%%%%%%%%%%%
\label{sec:BSM}

The discovery of the new boson in accordance with the Higgs boson predicted by the SM has been a major breakthrough in the contemporary particle physics. The Higgs boson mass is a free parameter in the SM and its measurement fixes all the other parameters related to the Higgs field, such as the coupling strengths with bosons and fermions. The current quest is to establish whether the properties of the discovered boson are consistent with the SM predictions, or it is only a component of a more entangled Higgs sector. Moreover, there are still several aspects that are not explained by the SM, such as the hierarchy problem or the nature of dark matter~\cite{Langacker:2010zza}.

Several theoretical models have been proposed to explain the deficiencies of the SM. One of the simplest extension of the SM Higgs sector requires the existence of an additional singlet scalar field, S, which is neutral under all quantum numbers of the SM gauge group~\cite{Robens:2015gla}. In general the singlet field mixes with the SM Higgs boson, H, allowing it to couple to the same states as the SM Higgs boson itself. If the mass of the scalar singlet was more than twice that of the SM Higgs boson, the S branching ratios would be reduced with respect to the H ones, because of the opening of the new $\mathrm{S \to HH}$ decay channel.

The mixing of the two states S and H would manifest as a suppression of the production cross section of both states and a suppression of the heavy mass Higgs boson decay modes to SM particles, if the $\mathrm{S \to HH}$ decay is kinematically accessible. In particular, identifying as H the observed Higgs boson with $m_\mathrm{H} = 125$\GeV, and supposing that the new scalar singlet S is heavier than H, one can introduce the scale factors of the low and high mass state couplings, $\mathcal{C}$ and $\mathcal{C'}$, respectively. These factors are bounded by the unitarity condition $\mathcal{C}^2 + \mathcal{C'}^2 = 1$. The singlet cross section and width are consequently modified by the factors $\mu'$ and $\Gamma'$, respectively:
\begin{equation}
\begin{split}
\mu' &= \mathcal{C'}^2 \cdot (1 - \mathcal{B}_\mathrm{new}) \quad ,\\
\Gamma' &= \Gamma_\mathrm{SM} \cdot \frac{\mathcal{C'}^2}{1 - \mathcal{B}_\mathrm{new}} \quad ,
\end{split}
\end{equation}

\noindent where $\mathcal{B}_\mathrm{new}$ is the singlet branching fraction to non-SM-like decay modes.

Other models, such as the \emph{two-Higgs-doublet model} (2HDM)~\cite{Branco:2011iw}, extend the minimal Higgs content requiring the introduction of a second Higgs doublet. 
The generalization of the SM Lagrangian with two complex scalar fields, which are $\mathrm{SU(2)_L}$ doublets, eventually gives rise to five physical Higgs bosons: a charged pair ($\mathrm{H^{\pm}}$); two neutral $CP$-even scalars (H and h, where $m_{\rm H}>m_{\rm h}$ by convention); and a neutral $CP$-odd scalar (A)~\cite{Craig:2013hca}. The parameter space of these 2HDM models can accommodate a wide range of variations in the production and decay modes of the SM-like Higgs boson. Nevertheless, tight constraints on flavour-changing neutral currents disfavour 2HDM with tree-level flavour violation. Similarly, limits on additional sources of $CP$ violation favour 2HDM with a $CP$-conserving potential. These assumptions significantly reduce the parameter space of 2HDM models. Moreover, if the h boson is identified with the observed 125\GeV boson, the experimental measurements further constraint the possible production and decay modes of the other predicted particles. Examples of possible decay channels in this framework are the following: the  heavy  $CP$-even  Higgs boson  may  decay  to  two  light  CP-even  Higgs bosons, $\mathrm{H \to hh}$; the $CP$-odd pseudoscalar Higgs boson may decay to a light $CP$-even Higgs and a Z boson, $\mathrm{A \to Zh}$; the charged Higgs bosons may decay to a SM-like Higgs and a $\mathrm{W}^\pm$ boson, $\mathrm{H^\pm \to \mathrm{W}^\pm h}$.

In order to search for new particles that could be ascribed to the simple models depicted above, or even to more complicated theories, it is of utmost importance to provide precise measurements of the Higgs boson couplings and kinematics, as well as its spin and parity properties. A complementary strategy is to perform direct searches for additional Higgs bosons in the full mass range accessible to current and future experiments.
