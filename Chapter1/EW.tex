\section{Electroweak interaction}
%%%%%%%%%%%%%%%%%%%%%%%%%%%%%%%%%%%%%%%%%%%%%%%%%%%%%%%%%%%%%%%%%%%%%%
\label{sec:EW}

The theory of electroweak interaction was formulated in the 1960s by S. L. Glashow, A. Salam and S. Weinberg~\cite{Glashow:1961tr,Weinberg:1967tq} as an $\mathrm{SU(2) \otimes U(1)}$ local gauge theory.
In 1957 the experiments lead by Madame Wu~\cite{Wu:1957my} and Garwin-Lederman-Weinrich~\cite{Garwin:1957hc} confirmed that the parity symmetry is maximally violated for weak charged current interactions, proving that only left-handed particles or right-handed antiparticles are involved in these interactions. 

To express this aspect the weak charged currents can be written making use of the Dirac spinors, $\ell$ and $\nu_\ell$ (representing the lepton and neutrino spinors, repsectively), as follows:

\begin{equation}
\begin{split}
J_\mu^+ &= \bar{\nu}_\ell \gamma_\mu \dfrac{1}{2}(1-\gamma^5) \ell \quad, \\
J_\mu^- &= \bar{\ell} \gamma_\mu \dfrac{1}{2}(1-\gamma^5)\nu_\ell \quad,
\end{split}
\end{equation}

where the ``$+$'' and ``$-$'' superscripts indicate the charge-raising and charge-lowering character of the currents, and $\gamma^5 = i\gamma^0\gamma^1\gamma^2\gamma^3$ is the product of the Dirac $\gamma$ matrices. In fact, the effect of the projector $P=\dfrac{1}{2}(1-\gamma^5)$ is to project the spinor to its left-handed component. Therefore, introducing the following doublet:

\begin{equation}
\chi_L = \begin{pmatrix} \nu_\ell \\ \ell \end{pmatrix}_L \quad ,
\end{equation}

where $L$ denotes the left-handed component of the spinor, and defining the ``step-up'' and ``step-down'' operators $\tau_\pm = \dfrac{1}{2}(\tau_1 \pm i\tau_2)$, where $\tau_i$ are the Pauli matrices, the charged currents become:

\begin{equation}
\begin{split}
J_\mu^+ &= \bar{\chi}_L \gamma_\mu \tau_+ \chi_L \quad, \\
J_\mu^- &= \bar{\chi}_L \gamma_\mu \tau_- \chi_L \quad.
\end{split}
\end{equation}

In order to complete the $\mathrm{SU(2)}$ invariance of the theory, a third conserved current should exist with the form:

\begin{equation}
J_\mu^3 = bar{\chi}_L \gamma_\mu \dfrac{1}{2} \tau_3 \chi_L = \dfrac{1}{2}\bar{\nu}_L \gamma_\mu \nu_L -  \dfrac{1}{2}\bar{\ell}_L \gamma_\mu \ell_L \quad,
\end{equation}

The three currents $J_\mu^\pm$ and $J_\mu^3$ constitute an isospin triplet of weak currents, with corresponding charges

\begin{equation}
T^i = \int J_0^i(x)d^3x \quad,
\end{equation}

that generate the $\mathrm{SU(2)}_L$ algebra defined by the following rules:

\begin{equation}
\left[T^i, T^j \right] = i \varepsilon_{ijk} T^k \quad.
\end{equation}

Nevertheless, $J_\mu^3$ cannot be identified with the weak neutral current, because the weak neutral current involves both left- and right-handed components. The electromagnetic current cannot be represented by $J_\mu^3$ as well, for the aforementioned reasons and because it cannot be coupled with the uncharged neutrino.

In order to save the $\mathrm{SU(2)}$ symmetry, the existence of a new $\mathrm{U(1)}$ symmetry is required, and a new conserved current arises. The new symmetry is known as hypercharge symmetry ($\mathrm{U(1)_Y}$) and the corresponding conserved current is:

\begin{equation}
j_\mu^Y = \bar{\psi} \gamma_\mu Y \psi \quad,
\end{equation}

where $\psi$ is a generic Dirac spinor and the hypercharge $Y$ is defined as:

\begin{equation}
Y = 2(Q - T_3)
\end{equation}

The $j_\mu^Y$ current is unchanged under $\mathrm{SU(2)}_L$ transformations (is an isospin singlet). The electromagnetic and weak interactions can be incorporated defining the electromagnetic current as:

\begin{equation}
j_\mu^{em} = J_\mu^3 + \dfrac{1}{2} j_\mu^Y \quad,
\end{equation}

which represents the electroweak unification. 
\begin{comment}
Also, the weak neutral current $J_\mu^{NC}$ can be written as a combination of $J_\mu^3$ and $j_\mu^{em}$ as follows:

\begin{equation}
J_\mu^{NC} = J_\mu^3 - \sin ^2 \theta_W j_\mu^{em} \quad,
\end{equation}

where $\theta_W$ is the Weinberg angle.
\end{comment}
The electroweak lagrangian can be expressed in a local $\mathrm{SU(2)}_L \otimes \mathrm{U(1)}_Y$ gauge invariant form by introducing the covariant derivatives $\mathcal{D}_\mu$ in place of the ordinary derivatives:

\begin{equation}
\mathcal{D}_\mu = \partial_\mu + i g \vec{A}_\mu \cdot \frac{\vec{\tau}}{2} - \frac{1}{2}i g' Y B_\mu \quad ,
\end{equation}

where $\vec{A}_\mu$ is a vector of three gauge fields satisfying the local $\mathrm{SU(2)}$ symmetry, and $B_\mu$ is the gauge field assuring the $\mathrm{U(1)}$ symmetry. The parameters $g$ and $g'$ represent the coupling constants for the gauge fields. The electroweak lagrangian can thus be written as:

\begin{equation}
\mathcal{L} = \sum_{f} \bar{\psi}i\gamma_\mu\mathcal{D}_\mu\psi = \sum_{f} \bar{\psi}i\gamma_\mu\partial_\mu\psi + \mathcal{L}_int \quad,
\end{equation}

where $\mathcal{L}_int$ represents the interaction terms.
