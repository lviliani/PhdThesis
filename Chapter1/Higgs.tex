\section{The Higgs mechanism}
%%%%%%%%%%%%%%%%%%%%%%%%%%%%%%%%%%%%%%%%%%%%%%%%%%%%%%%%%%%%%%%%%%%%%%
\label{sec:Higgs}

Local gauge symmetries naturally lead to the presence of gauge bosons, the exchange particles mediators of the fundamental interactions. The symmetry requires these gauge bosons to be massless, which is unproblematic for photons and gluons, but in drastic contrast to the known masses of the Z and $\mathrm{W^\pm}$ bosons, which are $m_\mathrm{Z} = 91.1876 \pm 0.0021$\GeV and $m_\mathrm{W} = 80.385 \pm 0.015$\GeV, respectively. Moreover, the maximally parity violating structure of the weak charged currents also breaks local gauge invariance for all massive fermions, due to their coupling to the W boson. This leads to the apparent antagonism that, while the $\mathrm{SU(2)_L \otimes U(1)_Y$ gauge symmetry does describe the coupling structure of the electroweak force, at the same time it seems to contradict the fact that the W and Z bosons, and all fermions have a non-vanishing mass. 

The proposed solution to this problem is the mechanism of spontaneous symmetry breaking, where the gauge symmetry is still intrinsic to the Lagrangian density of the theory, but not manifest in its energy ground state, which in this case is the quantum vacuum.
