\section{The strong interaction}
%%%%%%%%%%%%%%%%%%%%%%%%%%%%%%%%%%%%%%%%%%%%%%%%%%%%%%%%%%%%%%%%%%%%%%
\label{sec:QCD}

Quantum Chromo-Dynamics (QCD) is the theory that describes the strong interactions~\cite{Ellis:1991qj}. It is an unbroken gauge non-abelian theory based on the group $\mathrm{SU(3)}$ of colour ($\mathrm{SU(3)_c}$). The mediators of the interaction are eight massless gluons and the elementary particles of matter are colour triplets of quarks, with different flavours. In fact, as shown in Eq.~\eqref{eq:quarks}, six types (flavours) of quark exist and each quark possesses a colour charge that can assume three values, namely red, green and blue.

The physical vertices in QCD include the gluon-quark-antiquark vertex, analogous to the QED photon-fermion-antifermion coupling, but also the three-gluon and four-gluon vertices, i.e. gluon themselves carry colour charge, which have no analogue in an abelian theory like QED. Quark and gluons are the only particles that interact through the strong interaction.

The non-abelian nature of the theory leads to two important characteristics:
\begin{itemize}
\item \emph{colour confinement}: the QCD coupling constant $\alpha_s = g_s^2/4\pi$ is a function of the scale of the interaction $Q$. At low energy (corresponding to large distances of the order of 1\,fm) the $\alpha_s$ value is large and a perturbative approach is not applicable. When a quark-antiquark pair begins to separate, the colour field generated by the exchanged gluons increases its intensity and, at some point, becomes more energetically favourable to create a new quark-antiquark pair from the vacuum than increasing further the interaction strength. This explains why free quarks are not observed and the final state particles are made of colourless quark bound states (hadrons). This is also the cause of the hadronization process which causes the formation of jets.

\item \emph{asymptotic freedom}: the coupling constant decreases at large scales $Q$ approaching to zero, meaning that quarks can be asymptotically considered as free particles. The small value of the coupling constant at large scales justifies the usage of a perturbative approach to describe hard processes.
\end{itemize}



\subsection{Proton-proton interactions}

The fundamental difference between hadron and lepton collisions is the fact that hadrons, differently from leptons, are not elementary particles but have an internal structure, which can be described in terms of the QCD-improved parton model. The basic idea of this model is to represent the inelastic scattering as quasi-free scattering of point-like constituents within the proton, the partons~\cite{Altarelli:2013tya}. Hadrons, along with the valence quarks that contribute to their quantum numbers ($uud$ for protons), contain virtual quark-antiquark pairs known as sea quarks. Sea quarks arise from gluon splitting; a pair of quarks can in turn annihilate producing a gluon. In addition, gluons are present in the sea also owing to the three-gluon and four-gluon vertices.

In proton-proton collisions the interaction generally involves a pair of partons and any of the partons in the sea can interact with a given likelihood, making possible several types of interaction, such as $qq$, $qq'$, $q\bar{q}$, $q\bar{q}'$, $gq$, $g\bar{q}$ or $gg$.

At a hadron collider the partons entering the hard scattering carry an event-by-event variable fraction $x$ of the proton four-momentum, also known as Bjorken's scaling variable. Therefore the centre-of-mass energy of the hard scattering is given by $\sqrt{\hat{s}} = \sqrt{x_1 x_2 s}$, where $\sqrt{s}$ is the centre-of-mass energy of the incoming protons, and $x_1$, $x_2$ are the four-momentum fractions carried by the two interacting partons\footnote{Considering $\sqrt{s}=14$\TeV and $x_1,x_2 \approx 0.15 \mbox{--} 0.20$, the partonic centre-of-mass energy is of the order of 1--2\TeV.}. Since generally $x_1$ and $x_2$ have different values, the centre-of-mass frame of the interaction is boosted along the beam direction. While this represents an experimental difficulty, on the other hand it allows to explore a wider range of energies with respect to an electron-positron collider.

In order to evaluate cross sections in hadron collisions, the calculation can be factorized into long-distance and short-distance components according to the QCD factorization theorem~\cite{Collins:1989gx}. Therefore, a typical cross section calculation for an inclusive process $pp \to X$ consists of a term that describes the partonic hard scattering, which can be calculated using perturbative QCD, and factors that describe the incoming flux of partons, the \emph{parton distribution functions} (PDF) $f_{i}$, as shown in the following equation~\cite{Butterworth:2012fj}:

\begin{equation}
\sigma(pp \to X) = \sum_{i,j} \int dx_1 dx_2 f_{i}(x_1,\mu_F^2)f_{j}(x_2,\mu_F^2)\hat{\sigma}_{ij\to X}(x_1 x_2 s, \mu_R^2, \mu_F^2) \quad .
\end{equation}

In this expression the sum runs over all the initial-state partons with longitudinal momentum fractions $x_1$ and $x_2$, where the subscripts $1$ and $2$ refers to the two incoming protons. The factorization scale $\mu_F$ is an arbitrary parameter that represents the scale at which the separation between the hard perturbative interaction and the long distance, non-perturbative, evolution of the produced partons occurs. The $\hat{\sigma}_{ij\to X}$ term corresponds to the partonic cross section evaluated at the scales $\mu_F$ and $\mu_R$, where $\mu_R$ is the renormalization scale, an additional scale introduced in perturbative QCD to treat the ultraviolet divergences. The PDF $f_{i,j}$ represents the probability density for a parton $i,j$ to be found within the incoming proton and to carry a fraction $x_{1,2}$ of its momentum. The PDFs are obtained performing global fits to data at different scales $Q^2$ and their evolution with scale is governed by the DGLAP equation~\cite{Altarelli:1977zs}. The global PDFs fits are provided by three main collaborations: CTEQ~\cite{Nadolsky:2008zw}, MSTW 2008~\cite{Watt:2011kp} and NNPDF~\cite{Ball:2013hta}.



\subsection{Hadron collider kinematics}\label{sec:pp_kin}

As described before, at hadron colliders the centre-of-mass energy of the parton hard scattering is generally boosted along the beam direction. It is therefore useful to describe the final state in terms of variables that are invariant under Lorentz transformations along that direction. A convenient set of kinematic variables is the transverse momentum \pt, the rapidity $y$ and the azimuthal angle $\phi$. In term of these variables, the four-momentum of a particle of mass $m$ can be written as:

\begin{equation}
p^\mu = (E, p_x, p_y, p_z) = (m_\mathrm{T}\cosh y, p_\mathrm{T}\sin\phi, p_\mathrm{T}\cos\phi, m_\mathrm{T}\sinh y) \quad ,
\end{equation}

where $p_x$, $p_y$ and $p_z$ are the components of the momentum $\vec{p}$ ($p_z$ is directed along the beam direction) and the transverse mass is defined as $m_\mathrm{T} = \sqrt{\pt^2 + m^2}$. The rapidity $y$ is defined by the following formula:

\begin{equation}
y = \frac{1}{2} \ln \left(  \frac{E+p_z}{E-p_z} \right) \quad .
\end{equation}

The rapidity is not invariant under boosts along the beam direction but it transforms according to the law:

\begin{equation}
y \longrightarrow y + \frac{1}{2} \ln \left(  \frac{1+\beta}{1-\beta} \right) \quad ,
\end{equation}

where $\beta$ is the boost velocity. According to this definition the rapidity differences $\Delta y$ are Lorentz invariant. Experimentally it is more convenient to use the pseudorapidity $\eta$, defined as:

\begin{equation}
\eta = - \ln \tan \frac{\theta}{2} \quad ,
\end{equation}

where $\theta$ is the polar angle between the particle momentum and the beam direction ($\cos \theta = p_z/|\vec{p}|$). For ultra-relativistic particles the pseudorapidity coincides with the rapidity.








