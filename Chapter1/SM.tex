\section{The Standard Model of particle physics}
%%%%%%%%%%%%%%%%%%%%%%%%%%%%%%%%%%%%%%%%%%%%%%%%%%%%%%%%%%%%%%%%%%%%%%
\label{sec:SM}

The Standard Model of particle physics is the theory that describes all fundamental constituents of matter and their interactions~\cite{Halzen:1984mc}. It is a renormalisable quantum field theory based on a $\mathrm{SU(3)_c \otimes SU(2)_L \otimes U(1)_Y}$ local gauge symmetry, and is capable to provide a quantitative description of three of the four interactions in nature: electromagnetism, weak interaction and strong nuclear force. 

According to the SM, the ordinary matter is made up of spin-$1/2$ particles, denoted as fermions. The fermions are subdivided into two classifications of elementary particles: leptons and quarks. Both classes consist of six particles, grouped into three doublets, called generations. Three additional doublets for each class are composed of leptons and quarks antiparticles. A charged particle with electric charge $Q=-1$, either the electron (e), the muon ($\mu$) or the tauon ($\tau$), and a neutral particle (the corresponding neutrino), compose the following lepton generations, ordered according to an increasing mass hierarchy:
\begin{equation}
\label{eq:leptons}
\begin{pmatrix} \rm e^-       \\ \nu_{\rm e}      \end{pmatrix}, \quad
\begin{pmatrix} \mu^-     \\ \nu_{\mu}  \end{pmatrix}, \quad
\begin{pmatrix} \tau^-    \\ \nu_{\tau} \end{pmatrix}  \quad .
\end{equation}

Charged leptons can interact via the electromagnetic and weak force, while neutrinos, that are assumed to be massless, can interact only through the weak interaction.

Similarly, quarks are organized in pairs composed of a particle with $Q=+2/3$, which can be either the \emph{up} (u), \emph{charm} (c) and \emph{top} (t) quark, and another particle with $Q=-1/3$, the \emph{down} (d), \emph{strange} (s) and \emph{bottom} (b) quark:
\begin{equation}
\label{eq:quarks}
\begin{pmatrix} \rm u       \\ \rm d      \end{pmatrix}, \quad
\begin{pmatrix} \rm c       \\ \rm s      \end{pmatrix}, \quad
\begin{pmatrix} \rm t       \\ \rm b      \end{pmatrix}  \quad .
\end{equation}

As well as leptons, quarks can interact via the electromagnetic and weak force, but also through the strong interaction, responsible of their confinement within hadrons. In fact, free quarks are not observed in nature, but they bind together forming two categories of hadrons: mesons, bound states of a quark q and an anti-quark $\mathrm{\bar{q}}$, and baryons, bound states of three quarks.

In the SM the interaction between elementary particles occurs through the exchange of spin-1 particles, known as bosons, which identify the fundamental forces. The photon $\gamma$ is the mediator of the electromagnetic interaction, the $\mathrm{W^{\pm}}$ and Z bosons are the mediators of the weak interaction, while the strong force is mediated by eight gluons (g). Electromagnetic and weak interactions are actually the manifestations of the same fundamental interaction, the electroweak force.


%%% EWK
\subsection{The electroweak interaction}

The theory of electroweak interaction was formulated in the 1960s by S. L. Glashow, A. Salam and S. Weinberg~\cite{Glashow:1961tr,Weinberg:1967tq} as an $\mathrm{SU(2) \otimes U(1)}$ local gauge theory.
The Lagrangian density governing the electroweak interaction is therefore invariant under gauge transformations of the $\mathrm{SU(2)_L\otimes U(1)_Y}$ symmetry group. The $\mathrm{SU(2)_L}$ group refers to the weak isospin charge $I$, while $\mathrm{U(1)_Y}$ to the weak hypercharge $Y$, which are connected to the charge Q by the following equation:
\begin{equation}
Y = 2(Q - I_3) \quad,
\end{equation}

\noindent where $I_3$ represents the third component of the weak isospin. 

According to the Noether theorem, the $\mathrm{SU(2)_L}$ invariance of the theory leads to the existence of three conserved currents, $J_\mu^\pm$ and $J_\mu^3$, which constitute an isospin triplet of weak currents. 
The two currents $J_\mu^\pm$ represent the weak charged current interactions, which describe the interaction between fermions mediated by the charged $\mathrm{W^\pm}$ bosons. These currents only involve left-handed particles or right-handed anti-particles, in accordance with the fact that the parity symmetry is maximally violated for weak charged current interactions, as confirmed by the experiments of Wu~\cite{Wu:1957my} and Garwin-Lederman-Weinrich~\cite{Garwin:1957hc} in 1957. In the case of leptons, charged currents can only connect two particles within the same generation, for example the electron and the electron neutrino, while a mixing of different generations may occur in the case of quarks, according to the Cabibbo-Kobayashi-Maskawa matrix (CKM).

Another possible interaction in the weak sector is known as neutral current interaction and is mediated by the neutral Z boson. In the vertex of this interaction the identity of the interacting leptons does not change, resembling in this matter the electromagnetic current. Concerning the quark sector, the weak neutral currents involving different quark flavours, i.e. \emph{flavour changing neutral currents}, are strictly suppressed at tree level by the Glashow-Iliopoulos-Maiani (GIM) mechanism~\cite{Glashow:1970gm}.

Nevertheless, the other component of the weak isospin triplet of currents $J_\mu^3$, cannot be identified with the weak neutral current, because the latter involves both left- and right-handed components. The electromagnetic current cannot be represented by $J_\mu^3$ as well, for the aforementioned reason and because it cannot be coupled with the uncharged neutrino. In order to save the $\mathrm{SU(2)}$ symmetry, the existence of the $\mathrm{U(1)_Y}$ symmetry is required, and a new conserved current, $j_\mu^Y$, arises. The $j_\mu^Y$ current is unchanged under $\mathrm{SU(2)}_L$ transformations (is an isospin singlet) and is incorporated, together with $J_\mu^3$, in the definition of the electromagnetic current, giving rise to the electroweak unification.

Local gauge symmetries naturally lead to the presence of gauge bosons, the exchange particles mediators of the fundamental interactions. The symmetry requires these gauge bosons to be massless, which is unproblematic for photons and gluons, but in drastic contrast to the known masses of the Z and $\mathrm{W^\pm}$ bosons, which are $m_\mathrm{Z} = 91.1876 \pm 0.0021$\GeV and $m_\mathrm{W} = 80.385 \pm 0.015$\GeV, respectively. Moreover, the maximally parity violating structure of the weak charged currents also breaks local gauge invariance for all massive fermions, due to their coupling to the W boson. This leads to the apparent antagonism that, while the $\mathrm{SU(2)_L \otimes U(1)_Y}$ gauge symmetry does describe the coupling structure of the electroweak force, at the same time it seems to contradict the fact that the W and Z bosons, and all fermions have a nonvanishing mass. 

The proposed solution to this problem is the mechanism of \emph{spontaneous symmetry breaking}, where the gauge symmetry is still intrinsic to the Lagrangian density of the theory, but not manifest in its energy ground state, which in this case is the quantum vacuum. The spontaneous symmetry breaking of the $\mathrm{SU(2)_L \otimes U(1)_Y}$ symmetry group requires the introduction of a self-interacting complex scalar field~\cite{Wolf:2015kua}, which is an isospin doublet:
\begin{equation}
\phi = \begin{pmatrix} \phi^+       \\ \phi^0      \end{pmatrix} = \begin{pmatrix} (\phi_1+i\phi_2)/\sqrt{2}       \\ (\phi_3+i\phi_4)/\sqrt{2}      \end{pmatrix} \quad .
\end{equation}

\noindent The simplest lagrangian involving this field has the form:
\begin{equation}\label{eq:higgsL}
\begin{split}
\mathcal{L}_\mathrm{H} &= \mathcal{D}_\mu \phi^\dagger \mathcal{D}^\mu \phi - V(\phi) \quad,\\
V(\phi) &= -\mu^2\phi^\dagger\phi + \lambda(\phi^\dagger\phi)^2
\end{split}
\end{equation}

\noindent Here $\mathcal{D}_\mu$ represents the covariant derivative, which is defined as:
\begin{equation}
\mathcal{D}_\mu = \partial_\mu + i g \vec{A}_\mu \cdot \frac{\vec{\tau}}{2} - \frac{1}{2}i g' Y B_\mu \quad ,
\end{equation}
\noindent where $\vec{A}_\mu$ is a vector of three gauge fields satisfying the local $\mathrm{SU(2)}$ symmetry, and $B_\mu$ is the gauge field assuring the $\mathrm{U(1)}$ symmetry. The parameters $g$ and $g'$ represent the coupling constants for the gauge fields and $\vec{\tau}$ are the Pauli matrices.

The term $V(\phi)$ in Eq.\eqref{eq:higgsL} is a potential term that depends on two parameters, $\mu$ and $\lambda$, with $\lambda>0$ in order to have vacuum stability. If the $\mu$ parameter is chosen so that $\mu^2<0$, the symmetry of $V(\phi)$ may be broken, since the minimum of the potential may assume infinite values:
\begin{equation}
\phi^\dagger\phi = -\frac{\mu^2}{2\lambda} = \frac{v^2}{2} \quad ,
\end{equation}

\noindent where $v$ corresponds to the \emph{vacuum expectation value} (VEV). Perturbation theory requires an expansion of $\phi$ around its energy ground state. The ground state is chosen in such a way it  breaks the $\mathrm{SU(2)_L \otimes U(1)_Y}$ symmetry group but preserves the invariance under $\mathrm{U(1)_{em}}$ transformations, i.e. it has a null electric charge. This latter requirement guarantees the presence of a neutral massless gauge boson, the photon. Therefore, the ground state can be written without any loss of generality as:
\begin{equation}
\tilde{\phi} = \frac{1}{\sqrt{2}} \begin{pmatrix} 0 \\ v   \end{pmatrix}\quad .
\end{equation}
The field $\phi$ can be expanded at first order around the ground state obtaining:
\begin{equation}
\phi = \frac{1}{\sqrt{2}} \begin{pmatrix} 0 \\ v+h   \end{pmatrix}\quad .
\end{equation}
Introducing this field in the Higgs Lagrangian in Eq.~\eqref{eq:higgsL}, the bosonic fields acquire a mass given by:
\begin{equation}
m_\mathrm{W} = \frac{v}{2}g \quad, \qquad m_\mathrm{Z} = \frac{v}{2}\sqrt{g^2 +g'^2} \quad.
\end{equation}

Furthermore, given the self-interaction terms of the $h$ field, a new physical state (the Higgs boson) also arises, with a mass given by:
\begin{equation}
m_\mathrm{H} = v\sqrt{2\lambda} \quad,
\end{equation}
\noindent whose value is not predicted by the theory, since $\lambda$ is unknown\footnote{On the other hand, the value of $v$ can be obtained using the relation between $m_\mathrm{W}$ and the Fermi constant $G_\mathrm{F}$, which leads to $v = 1/\sqrt{\sqrt{2}G_\mathrm{F}} = 246.22$\GeV, setting the scale of the electroweak symmetry breaking.}.

The mass of fermions is achieved without breaking the gauge symmetry of the Lagrangian by introducing a coupling term, known as Yukawa coupling, between the fermion doublets and the Higgs field.

In addition to the mass of the particles, the model also predicts the couplings $f$ of the Higgs boson to fermions and heavy gauge bosons, despite their numerical values need to be determined by experiments:
\begin{equation}
\begin{split}
f_\mathrm{H\to ff} \propto \frac{m_\mathrm{f}}{v} &,\quad \text{fermions}\\
f_\mathrm{H\to VV} \propto \frac{2m_\mathrm{V}^2}{v} &,\quad \text{heavy bosons trilinear}\\
f_\mathrm{HH\to VV} \propto \frac{2m_\mathrm{V}^2}{v^2} &,\quad \text{heavy bosons quartic}\\
f_\mathrm{H\to HH} \propto \frac{3m_\mathrm{H}^2}{v} &,\quad \text{Higgs boson trilinear}\\
f_\mathrm{HH\to HH} \propto \frac{3m_\mathrm{H}^2}{v^2} &,\quad \text{Higgs boson quartic} \quad .
\end{split}
\end{equation}

During the past decades the predictions of the SM have been confirmed by experimental results with outstanding precision, and in 2012 the discovery of a new boson with a mass of about 125\GeV, consistent with the predicted Higgs boson, was announced by the ATLAS and CMS experiments at LHC.


\subsection{The strong interaction}

Quantum Chromo-Dynamics (QCD) is the theory that describes the strong interactions~\cite{Ellis:1991qj}. It is an unbroken gauge non-abelian theory based on the group $\mathrm{SU(3)}$ of colour ($\mathrm{SU(3)_c}$). The mediators of the interaction are eight massless gluons and the elementary particles of matter are colour triplets of quarks. As shown in \eqref{eq:quarks}, six types (flavours) of quark exist and each quark possesses a colour charge that can assume three values, namely red, green and blue.

The physical vertices in QCD include the gluon-quark-antiquark vertex, analogous to the Quantum Electro-Dynamics (QED) photon-fermion-antifermion coupling, but also the three-gluon and four-gluon vertices, i.e. gluon themselves carry colour charge, which has no analogue in an abelian theory like QED. Quarks and gluons are the only particles that interact through the strong interaction.

The non-abelian nature of the theory leads to two important characteristics:
\begin{itemize}
\item \emph{colour confinement}: the QCD coupling constant $\alpha_s = g_s^2/4\pi$ is a function of the scale of the interaction $Q$. At low energy (corresponding to large distances of the order of 1\,fm) the $\alpha_s$ value is large and a perturbative approach is not applicable. When a quark-antiquark pair begins to separate, the colour field generated by the exchanged gluons increases its intensity and, at some point, the creation of a new quark-antiquark pair from the vacuum becomes more energetically favourable than increasing further the interaction strength. This explains why free quarks are not observed and the final state particles are made of colourless quark bound states (hadrons). This is also the origin of the hadronization process which causes the formation of jets.

\item \emph{asymptotic freedom}: the coupling constant decreases at large scales $Q$ approaching to zero, meaning that quarks can be asymptotically considered as free particles. The small value of the coupling constant at large scales justifies the usage of a perturbative approach to describe hard processes.
\end{itemize}

