\section{The Standard Model of particle physics}
%%%%%%%%%%%%%%%%%%%%%%%%%%%%%%%%%%%%%%%%%%%%%%%%%%%%%%%%%%%%%%%%%%%%%%
\label{sec:SM}

The Standard Model of Particle Physics (SM) is the theory that describes all fundamental constituents of matter and their interactions. It is a quantum field theory based on a $\mathrm{SU(3)_c \otimes SU(2)_L \otimes U(1)_Y}$ local gauge symmetry, and is capable to provide a quantitative description of three of the four interactions in nature: electromagnetism, weak interaction and strong nuclear force. During the past decades the predictions of the SM have been confirmed by experimental results with outstanding precision.

The SM can be divided in two sectors: the electroweak sector and the strong sector, known as Quantum Chromo-Dynamics (QCD). The aforementioned symmetry holds only if the fermion fields are massless, in contrast with the experimental observations of massive fermions. A mechanism, known as \emph{spontaneous symmetry breaking}, is introduced in the SM allowing the elementary particles to acquire mass. This mechanism requires the presence of a new field, known as Higgs field.
