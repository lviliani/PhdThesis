\section{Event reconstruction and particle identification}
%%%%%%%%%%%%%%%%%%%%%%%%%%%%%%%%%%%%%%%%%%%%%%%%%%%%%%
\label{sec:Objects}

In CMS, the physics object reconstruction and identification is based on standard algorithms developed by the collaboration and used by all the physics analyses. In this section, the techniques used for the reconstruction and identification of the physics objects of interest for \hwwllnn analyses are described.

\subsection{The Particle Flow technique}

The Particle Flow (PF) event reconstruction technique~\cite{CMS-PAS-PFT-09-001} aims at the reconstruction and identification of all the stable particles in the event, i.e. electrons, muons, photons, charged and neutral hadrons, with a thorough combination of the information from all CMS sub-detectors, in order to determine their energy, direction and type. These individual particles are then used, for example, to build jets, to measure the missing transverse energy \MET, to reconstruct the $\tau$ from their decay products, to quantify the charged lepton isolation and to tag b-jets.

The CMS detector is well suited for this purpose. Indeed, the presence of a large internal silicon tracker immersed in an intense solenoidal magnetic field allows the reconstruction of charged particles with high efficiency and small fake rate, and provides a high precision measurement of the particle \pt down to about $150\,\MeV$, for $|\eta|\leq2.6$. The high granularity of the ECAL calorimeter is the additional key element for the feasibility of the PF technique, allowing the reconstruction of photons and electrons with high energy resolution.

The first step of the PF technique consists in the reconstruction of the basic elements from the various sub-detectors, such as charged-particle tracks, calorimeter clusters and muon tracks. These elements, which are provided by the sub-detectors with high efficiency an low fake rate, are then connected together with a link algorithm.

The good performance of the tracking system are achieved by means of an iterative tracking strategy~\cite{Chatrchyan:2014fea}, based on the Kalman Filter algorithm~\cite{Billoir:1990we}. The basic idea of iterative tracking is that initial iterations search for tracks that are easiest to find, e.g. high \pt tracks produced near the interaction region. After each iteration, hits associated to reconstructed tracks are removed from the hit collection, thereby reducing the combinatorial complexity and simplifying the subsequent iterations, which aim at finding more complicated set of tracks, e.g. low \pt or displaced tracks. The \emph{Iteration 0}, where the majority of tracks are reconstructed, is designed to identify prompt tracks with $\pt>0.8$\,\GeV that have three hits in the three layers of the pixel detector. \emph{Iteration 1} is used to recover prompt tracks that have only two pixel hits. \emph{Iteration 2} aims at finding low-\pt prompt tracks while \emph{Iterations 3--5} are intended to find tracks that originate outside the collision point, i.e. tracks produced by a secondary vertex, and to recover undetected tracks in the previous iterations. Each iteration proceeds according to four steps:
\begin{itemize}
\item \emph{seeding}: initial track candidates are obtained using 2 or 3 hits in the innermost layers (these proto-tracks are called seeds);
\item \emph{pattern recognition}: this step is based on Kalman Filter and searches for hits in the outer layers that could be associated to the initial track candidate, reconstructing the particle trajectory;
\item \emph{track fitting}: in this step a fit of the trajectory is performed, using its associated hits and providing an estimate of the track parameters (\pt, $\eta$, $\phi$, charge, etc.);
\item \emph{selection}: finally tracks are selected based on quality requirements.
\end{itemize}

The high detection efficiency of the calorimeters is based on a specific calorimeter clustering algorithm, which is performed separately in each sub-detector. The algorithm is based on three steps: guarda PFT-09-001

\subsection{Leptons reconstruction and identification}

	\subsubsection{Electrons}
	
	\subsubsection{Muons}
	
\subsection{Jets reconstruction and identification}

	\subsubsection{Jet b tagging}
