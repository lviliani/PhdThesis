\section{Tracking}
%%%%%%%%%%%%%%%%%%%%%%%%%%%%%%%%%%%%%%%%%%%%%%%%%%%%%%
\label{sec:tracking}

Tracking in CMS is an iterative procedure based on the Kalman filter. The basic idea of iterative tracking is that initial iterations search for tracks that are easiest to find, e.g. high \pt tracks produced near the interaction region. After each iteration, hits associated to reconstructed tracks are removed from the hits collection, thereby reducing the combinatorial complexity and simplifying the subsequent iterations, which aim at finding more complicated set of tracks, e.g. low \pt or displaced tracks. The \emph{Iteration 0}, where the majority of tracks is reconstructed, is designed to identify prompt tracks with $\pt>0.8$\,\GeV that have three hits in the three layers of the pixel detector. \emph{Iteration 1} is used to recover prompt tracks that have only two pixel hits. \emph{Iteration 2} aims at finding low-\pt prompt tracks while \emph{Iterations 3--5} are intended to find tracks that originate outside the collision point, i.e. tracks produced by a secondary vertex, and to recover undetected tracks in the previous iterations. The main differences between the six iterations lie in the configuration of the seed generation and the final track selection. 

Each iteration proceeds according to four steps:
\begin{itemize}
\item \emph{seeding}: initial track candidates are obtained using 2 or 3 hits in the innermost layers (these proto-tracks are called seeds);
\item \emph{pattern recognition}: this step is based on Kalman Filter and searches for hits in the outer layers that could be associated to the initial track candidate, reconstructing the particle trajectory;
\item \emph{track fitting}: in this step a fit of the trajectory is performed, using its associated hits and providing an estimate of the track parameters (\pt, $\eta$, $\phi$, etc.);
\item \emph{selection}: tracks are eventually selected based on quality requirements.
\end{itemize}

