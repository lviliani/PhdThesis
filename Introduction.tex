\chapter*{Introduction}
\addcontentsline{toc}{chapter}{Introduction}  
\markboth{Introduction}{}
\thispagestyle{empty}

The Higgs mechanism is an essential element of the Standard Model (SM), explaining the origin of mass and playing a key role in the physics of electroweak symmetry breaking. A suitable Higgs boson candidate, predicted by the Higgs mechanism, was found with a mass of 125\GeV by the ATLAS and CMS experiments in the first run of the CERN Large Hadron Collider (LHC).

After the discovery, the accurate measurement of the Higgs boson properties has become one of the main goals of the LHC. The Higgs sector could in fact be more entangled with respect to what discovered so far and Beyond the Standard Model (BSM) effects could emerge from accurate measurements of the couplings with fermions and bosons, which can be determined from the Higgs boson production processes and decays. The $\mathrm{H \to W^+W^-}$ channel is one of the most sensitive to these effects and the high branching fraction allows the statistics needed for a precision measurement to be collected.

Measurements of the production cross section of the Higgs boson times branching fraction of decay ($\sigma\times\mathcal{B}$) in a restricted part of the phase space (fiducial phase space) and its kinematic properties thus represent an important test for possible deviations from the SM predictions.
In particular, it has been shown that the Higgs boson transverse momentum spectrum can be significantly affected by the presence of physics phenomena not predicted by the SM. %In addition, these measurements facilitate tests of the theoretical calculations in the SM Higgs sector.

A complementary strategy to seek hints of BSM physics is to perform direct searches for additional Higgs bosons in the full mass range accessible to current and future experiments. There are indeed several models that predict a richer Higgs sector, requiring the existence of new particles that could show up in direct searches at energies achievable at LHC.

The purpose of this thesis is twofold, reporting firstly a measurement of the Higgs boson transverse momentum spectrum using proton-proton collision data collected at a centre-of-mass energy of 8\TeV, and thereafter focusing on the first data collected by the CMS experiment at the unprecedented centre-of-mass energy of 13\TeV. These latter data will be used both to perform a preliminary re-discovery measurement of the Higgs boson at the new collision energy, and to directly search for new resonances with masses up to 1\TeV. All the analyses discussed in this thesis are performed by selecting the Higgs boson (or the new resonance) decays to a W boson pair.

The transverse momentum spectrum and the inclusive $\sigma\times\mathcal{B}$ measurements reported here, represent the first measurements at the LHC using $\rm{H}\to\rm{W^+W^-}\to e^{\pm} \mu^{\mp}\nu\bar{\nu}$ decays.
The Higgs boson transverse momentum is experimentally reconstructed using the lepton pair transverse momentum and missing transverse momentum. The differential $\sigma\times\mathcal{B}$ is measured as a function of the Higgs boson transverse momentum in a fiducial phase space defined to match the experimental acceptance in terms of lepton kinematics and event topology. The measurements are corrected for detector effects taking advantage of an unfolding procedure, and compared to theoretical calculations based on the SM, to which they agree within experimental uncertainties.

The first data collected during 2015, corresponding to an integrated luminosity of 2.3\ifb, are used to perform a re-discovery analysis of the Higgs boson in the $\rm{H}\to\rm{W^+W^-}\to e^{\pm} \mu^{\mp}\nu\bar{\nu}$ decay channel. Since the amount of analysed data is small, this preliminary result does not provide a significant signal evidence, as expected due to the value of the Higgs boson production cross section predicted by the SM. Nevertheless, this analysis is useful to pave the way for similar analyses that are now studying the same channel with an enlarged data set.

The same data are used to search for new resonances decaying to $\rm{W^+W^-}\to e^{\pm} \mu^{\mp}\nu\bar{\nu}$ in the mass range between 200\GeV and 1\TeV. No significant excess with respect to the SM background expectation is observed, and exclusion limits on the resonance $\sigma\times\mathcal{B}$ are reported over the whole analysed mass spectrum, assuming different hypotheses of the resonance decay width.
